\chapter*{Preface}
\addtocontents{toc}{\protect\contentsline{section}{\protect Preface}{\thepage}} 
\addtocontents{toc}{\bigskip} 

\lhead[{\it\fontsize{9pt}{9pt}\selectfont\thepage}]{\it{\fontsize{9pt}{11pt}\selectfont Preface}}

Professor C.V\@. Raman was the most outstanding scientist that India gave the world. He not only became famous for his
work but put India on the scientific map. He was a very colourful personality and a brilliant spokesman for science in India. Despite the great disadvantages, namely the lack of a tradition for research in the country, the very little encouragement he received for pursuing a scientific career and the meagre equipment available, Raman rose to the pinnacle of science and eminence. He gave up a lucrative Government job, to take up a Professorship that offered a fraction of the salary, only to advance his career as a scientist. His reward was the Nobel Prize for physics that was awarded to him in 1930 for his discovery of the light-scattering effect named after him.

He was a towering and dynamic scientific leader who inspired generations of students he trained in the methodology of research. He created the school of physics in Calcutta first, and then in Bangalore. His disciples came from all parts of India. He kindled their scientific interest and curiosity, and communicated to them the scientific spirit and the joy of scientific research. Many of his students and associates became in their lifetimes leaders of science and, in turn, created their own schools of research.

Raman was a brilliant and powerful speaker and his lectures,\break often with demonstrations, were a treat to listen to. A listener could never forget the exciting way in which he expounded a subject and the enthusiasm he generated. He wrote his scientific  papers with meticulous care and often used Latin expressions to emphasise a point. His flow and style of English were extraordinarily lucid; reading his scientific memoirs is like reading a piece of literary exposition.

He founded several scientific journals and nurtured them with care, for he believed that the quality of work and the quality of a journal go together. He published most of his scientific papers in these journals and encouraged his students and coworkers to do the same. He founded a Science Academy in Bangalore to serve as a forum for discussing scientific results and disseminating scientific knowledge through meetings, lectures and publications, by the Fellows of the Academy and their associates. He took a keen interest in the affairs of the Academy and got promising young scientists elected to the Fellowship of the Academy, long before they reached the peaks in their career. He could recognise talent and merit instinctively.

After his formal retirement from the Indian Institute of Science he founded the Raman Research Institute to pursue his
scientific interests, for retirement from scientific work was inconceivable for \hbox{Raman}. The pursuit of Science was the most joyous experience for him; it was the breath of his life. His scientific interests spanned subjects ranging from physics to biology and he could be truly called a Natural Philosopher, a vanishing breed in these days of extreme\break specialisation.

Raman was fiercely independent in thought and action and fearless in expressing his views. He was respected and honoured by\break maharajahas, princes, politicians and the general public. The mention of the name Sir C.V\@. Raman evoked respect and admiration\break everywhere in India.

Raman had a finely honed aesthetic sense and loved Nature. He loved colour, wherever it was found, in trees, gardens, flowers, sunsets, mountains and lakes. He was very curious about Nature and\break natural phenomena. If a subject interested him, he went into it deeply without any preconceived ideas. He would question all previously held views and reject them if they proved contrary to his experience. He had the gift of being able to reduce complex problems to simple and fundamental propositions. He often used to remark that a\break researcher should get to the wood and not be lost in the foliage. \hbox{Raman} was an experimentalist {\em par excellence}. In the understanding of physical phenomena, his physical intuition often leaped several steps over mathematics.

Raman was kind and generous to his associates and students. He was quick to express appreciation of good work and also to give tremendous encouragement at the right time. But he was also a man of strong emotions and could get violently angry, when provoked. In life he had fought many battles, both scientific and nonscientific. He had to face very difficult situations, but he never gave up hope, rejuvenating himself by immersion in scientific work.

What motivated Raman to do the extraordinary things he did against all odds? What was the secret of his success? These are questions to which there can be no simple answers. In fact, there will be as many answers as there are minds that ponder over such questions. But the events and course of his life are there, as facts, to examine, to marvel at and, perhaps, to provide a glimpse of understanding.

I had the greatest good fortune to be associated with him for eleven years, from 1949 to 1960, and moved closely with him, on a day to day basis. He was very kind and generous to me and trusted me very much. In fact, he treated me like his son, sharing his views, his ideas and his dreams. For me, this was a period of great education and experience and I owe my whole scientific career to him.

There are only a few biographical sketches on Raman --- very few in fact --- including a recent one by G. Venkataraman. I thought a first-hand account of Raman and his life would be of interest to and, perhaps, inspire, some young aspirants to a career in Science. My account is, therefore, highly personalised, contains a lot of anecdotes and has been written with a lay reader (with some interest in Science) in mind. I have tried to cover the entire life of Raman to make this memoir complete. But because of my association with Raman in the later period, it may seem that I have devoted more pages to this period. My justification is that you get a more complete view of a person only from the vantage point of proximity. I, however, did not keep any diary and, hence, most of this account is from memory.

In writing this memoir, I have used material from the following: (1) An extraordinarily interesting and lucid account of Raman by the late Dr. L.A. Ramdas in two articles published in the {\em Indian Journal of Physics Education} in 1971; (2) A concise but authentic and accurate biographical sketch on Raman by Prof. S. Bhagavantam, published by the Andhra Pradesh Academy of Sciences, Hyderabad;\break (3) Extracts from the Indian Academy of Sciences' publication in 1984 entitled {\em The First Fifty Years}; (4) Extracts from Prof.\break S. Ramaseshan's articles and lecture, `C.V. Raman Memorial Lecture' given at the Indian Institute of Science, Bangalore, and from\break {\em C.V. Raman and the German Connection}, an article written on the occasion of the Silver Jubilee of the Max Mueller Bhavan; (5) Extracts from the {\em Calcutta Municipal Gazette} dated July 4, 1931;\break (6) Extracts from the article on Sommerfeld's meeting with Raman in Calcutta in 1928, published by Dr. G. Torkar in the {\em Journal of\break Raman Spectroscopy}, 1986; (7) Extracts from the article {\em Golden\break Jubilee of the Discovery of the Raman Effect}, February 28, 1978, by the late Prof. K.R. Ramanathan in the Andhra Pradesh Academy of Sciences, Hyderabad; (8) Extracts from {\em Bhavan's Journal}, December 1970; (9) Extracts from the {\em Pasadena Star}, very kindly supplied by Paula Agranat Hurwitz of the California Institute of Technology Archives, Robert A. Millikan Memorial Library; and (10) Prof. B.S. Ramakrishna's articles on Indian Musical Drums and on Raman in {\em Science Today}, 1970. All these sources have been invaluable to me and I wish to record my deep sense of gratitude to the authors of these articles and their publishers. My grateful thanks are also due to Professor S. Chandrasekhar for permission to quote some of his statements.

I am very indebted to Prof. A.K. Ramdas in many ways. He patiently read the manuscript and made illuminating comments and suggestions. He agreed to write a foreword and made available to me much interesting material concerning Raman from his father's writings and collection.

My associate Ralph G. Maines has been extremely helpful in the course of the preparation of the manuscript and I wish to record my thanks to him. I am very indebted to Mrs. Alyne E. Bonnell in our text processing centre, for the enthusiasm and readiness with which she did the typing and the keen interest she took in the work. My wife Kamala has been a constant source of encouragement to me throughout my scientific life. She read the manuscript of this memoir and made interesting comments from the point of view of a nonscientific reader, for which I wish to record my sincere thanks to her. This work would not have been possible without the support and encouragement of AT\&T Bell Laboratories, my employer for the last quarter century.

\bigskip

\hfill A. JAYARAMAN

\noindent
18th May 1989

\noindent
AT\&T Bell Laboratories

\noindent
Murray Hill, New Jersey-07974

\noindent
U.S.A.
