\chapter{}\label{chap2}
\addtocontents{toc}{\bigskip}
\addtocontents{toc}{\protect\contentsline{chapter}{\protect \centerline{Chapter \numberline{\thechapter}}}{}}  
\addtocontents{toc}{\medskip}

\heading{To Bangalore and the Indian Institute of Science}
\addtocontents{toc}{\protect\contentsline{section}{To Bangalore and the Indian Institute of Science}{\thepage}}
\smallskip
\index{Indian Institute of Science|(}

\lhead[{\it\fontsize{9pt}{9pt}\selectfont\thepage}]{\it{\fontsize{9pt}{11pt}\selectfont To Bangalore and the Indian Institute of Science}}

\noindent
In 1933, Raman was offered the Directorship of the Indian
Institute of Science (I.I.Sc.) founded by J.N. Tata.\index{Tata, J.N.} He accepted it with
some hesitation. But after 25 years of living in Calcutta, Raman
was naturally attracted to Bangalore for the welcome change it
offered, both from the point of view of the climate and the
beautiful environment. He fell in love with Bangalore so much,
he made it his permanent home.


The Indian Institute of Science was set up in 1909 to conduct
original research and provide advanced training in science and
engineering for Indian students. To J.N. Tata, the visionary, the
institution was to be the primary base for the intellectual
rejuvenation and modernisation of India. But to Lord Curzon,\index{Curzon, Lord}
the then Viceroy of India, this would be a seditious step against
the British Raj and so he opposed its establishment. Nevertheless,
five years after Tata's death, the Institute was established, and
Bangalore became its home because of the far-sightedness of the
Maharajah and the Government of Mysore who offered it about
150 hectares of land and many other facilities.

The Directors of the Institute (till Raman's appointment)
were always British; so were most of the early faculty. Some felt
that under the tutelage of the British Resident of Mysore State,
the Institute was subserving the interests of Great Britain. Even
the \hbox{starting} of the General Chemistry and Electrical Technology
Departments were associated, by those who held this view, with
the running of the British-owned Kolar Gold Fields. The effect
of research and education `appearing' at the I.I.Sc. had a
tremendous effect on the country, but nationalists however were
dissatisfied with the `performance' of the Institute. They felt that,
in spite of large sums of money being spent, the Institute neither
catalysed industrial growth nor produced any outstanding
scientific discoveries of which India could be proud.

Raman, however, felt strongly that research and advanced
education could be the foundation for any economic advancement
only if there was excellence of the highest order. Therefore he
tried many strategies to bring about change at the Indian Institute
of Science.

Being a great lover of Nature,\index{Raman, Chandrasekhara Venkata!Traits/Interests} he first sought to improve
the surroundings by planting beautiful flowering trees. In this
he was aided by Sir Mirza Ismail,\index{Ismail, Mirza, M.} the Dewan of Mysore, and
Mr. Krumbeigel,\index{Krumbeigel} the Chief Horticulturist of Lalbagh. Even today,
the campus of the Institute remains a lovely garden, though
somewhat spoiled by the addition of so many buildings.

Around the time Raman moved to Bangalore, many famous
scientists were fleeing Germany. Raman wanted to attract them.
He felt that by inviting some of these outstanding men of science
and offering them permanent positions, a great scientific
movement could be created in India. He invited Max Born\index{Born, Max|(} and
nearly succeeded in appointing him to an Extraordinary Chair
of Physics he created at the Institute. Raman also made an offer
to Schr\"odinger,\index{Schr\"odinger} but he was too late, for the latter had already
accepted an offer from Dublin. Many others were on his list, but
before he could pursue his quest further, his strong personality
and management style resulted in serious difficulties with the
governing body of the Institute and he was forced to resign from
the Directorship. He, however, continued as Professor of Physics
until his retirement in 1948, pursuing with undiminished
enthusiasm new avenues of research.

Raman developed the Department of Physics at the Institute
and, with characteristic energy, turned it into an active centre
for physics research. He took a large number of students and
trained them as first-class physicists, shaping their careers and
destinies. He started research in diverse fields, {\em viz}. ultrasonics,
Brillouin scattering, X-ray scattering, physics of the diamond and
lattice dynamics. Light-scattering studies continued as a regular
programme in his laboratory.

One of his outstanding contributions during his 15 years at
the Institute concerns the diffraction of light by high-frequency
sound-waves. The phenomenon of ultrasonic diffraction was first
discovered in 1932, by Debye\index{Debye} and Sears\index{Sears} in America and,
simultaneously and independently by Lucas\index{Lucas} and Biquard\index{Biquard} in
France. Raman explained this phenomenon in a beautiful way.
He also published\break \hbox{several} papers with Nagendra Nath\index{Nath, Nagendra} as his
collaborator. The so-called Raman-Nath theory not only fully
explained the observed effects but indicated several interesting
new ideas.

How Raman's brilliant suggestions led to the Raman-Nath
theory has been described by Nagendra Nath himself:
\begin{quote}
{\fontsize{10}{12}\selectfont
``One day, Parthasarathy was to give an account of the
determination of the velocity of sound in organic liquids by the
method of diffraction of light by ultrasonic waves. Hardly had
he finished the description of the experimental set-up, than
Professor raised the query --- What is the number of diffraction
orders expected on the basis of Brillouin's theory? --- The reply
was two first orders of weak intensity. What was the fact, was
the next query. A number of diffraction orders not in agreement
with the theory. Professor went to the board and said that the
theory should be developed in a different way. A sound wave
creates compressions and rarefactions. A light beam would be
slowed in the region of compressions and it would move faster
in the region of rarefactions, and so a plane wavefront would
become a corrugated wavefront like a zinc sheet used for building
purposes. Professor said that an analysis of this corrugated
wavefront would explain the unexplained results. When I went
to him next day giving an explanation of the results on the basis
of his ideas, he said it was all correct and that started a series
of papers by him and myself which has come to be known in
literature as the Raman-Nath Theory''.}\relax
\end{quote}

\newpage

Though Raman's outlook was essentially that of an
experimental physicist, he would insist on the physical significance
of every theoretical result. He had a stock in trade of certain
physical results and he would liberally draw on them to explain
results in a different subject altogether. Once, Max Born
exclaimed, ``He leaps over Mathematics''.

\index{Raman, Chandrasekhara Venkata!Traits/Interests|(}
It was in the field of lattice dynamics that Raman got
involved in a bitter controversy with Born, Debye\index{Debye} and others by
strongly opposing their theories. Raman was incorrect, but he
was convinced that he was right in his approach. This attitude
made him highly emotional and irrational when it came to lattice
dynamics. Further, it also proved counter-productive for him and
he got side-tracked into an area which was not his fort\'e. What
he was, was an experimentalist {\em par excellence}, possessing an
unusual insight when it came to optics.

Raman had a lively and life-long interest in diamonds and
built up an outstanding collection. He made many scientific
studies on the diamond, but unfortunately got into a controversy
again.

These controversies do not diminish Raman's fine
contribution made to optics, spectroscopy and crystal physics
during these years. Raman and Nedungadi\index{Nedungadi} were the first to
observe the soft mode behaviour associated with the phase
transition in the $\alpha$ to $\beta$ quartz transformation. This was almost
two decades before Cochran\index{Cochran} came out with the soft mode theory
for phase transitions, a theory now well-known. Raman and his
students also made outstanding contributions to crystal optics, {\em viz}.
conical refraction, optical activity and other crystal
optical phenomena.

During his tenure as Professor at the Indian Institute of
Science, the Department of Physics was intensely active in diverse
areas of physics, with Raman as its moving spirit. It was during
this period that R.S. Krishnan\index{Krishnan, R.S.} carried out pioneering Raman
scattering studies using the so-called Rasetti technique which
employed the mercury resonance radiation $\lambda$ = 2537\r{A} coming out
of a water-cooled quartz mercury lamp placed between the poles
of a magnet. This technique worked so well for substances
transparent to the above ultra-violet radiation that even second-order
 Raman spectra could be obtained. The second-order
spectrum is a thousand times weaker compared to the first-order
Raman scattering. The substances investigated included diamond,
quartz, rock salt, magnesium oxide etc. It was this work that
set Raman on his new lattice dynamics. After Raman's
retirement, Krishnan became Professor of Physics and continued
the tradition of Raman spectroscopy with his students and
co-workers.

\vskip .05cm

Raman's career at the Indian Institute of Science was not
altogether a happy experience for him. His forced resignation
from the Directorship and all the unpleasant incidents preceding
it made him very bitter. He often expressed his bitterness to
visitors. Why did Raman fail miserably? What really happened
that led to his resignation from the Directorship? These are
questions without any real answers. S. Ramaseshan,\index{Ramaseshan, S.} a former
Director of the I.I.Sc., has cited some of the views expressed in
this connection: Raman came to the Institute with the idea of
making it a centre of science of international standard. What he
found was a quiet sleepy place where little work was done by
a number of well-paid people... Raman's speeding up of the entire
pace of the Institute was bound to look like criticism of the former
management. He made the mistake of not waiting for a year or
two before starting actual reforms (Born).

\vskip .05cm

Raman was obviously surrounded by people, both British
and Indian, who largely looked upon I.I.Sc. as a source of sinecure
positions. Some of the English faculty resented working under
Raman --- an Indian --- an experience they had never had before.
These faculty members had gained the ear of the colonial
Government which agreed to put pressure on the Tata family.
The Tata representatives on the Council were particularly sensitive
to the colonial Government's wishes, as the House of Tatas was
very dependent on Government protection. ``There is no Indian
physicist of the rank of \hbox{Raman}. No man can compare with him
in regard to vigour or intensity. This European intensity which
Raman exhibited to a marked degree seemed to make many
Indians suspicious of him,'' Born has stated. All changes made
by Raman provoked resentment. Many even felt that physics was
in the process of becoming a dominant feature of the Institute.
Raman, far too conscious of his own superiority, made other
people feel small in his presence. His acute mind and sharp tongue
seemed to constantly provoke resentment and tension around him.

All these could have been contributing factors. For one
thing, a man of success in India is an object of great envy and
he often finds himself in situations full of intrigues and plotting.
In fact, Raman's leaving Calcutta appears to have been the result
of intrigues and backbiting created by jealous colleagues. The
Indian Institute of Science was a much bigger institution even
in those times, with a number of scientists and engineers with
widely different backgrounds and loyalties. Such being the case,
it would have been a difficult task for even a very tactful person
to run the place smoothly. Unfortunately, Raman was a person
who had no patience, was far from tactful and did not have much
tolerance for different viewpoints. He was easily excitable and
said what he thought of people and their actions straight to their
faces. Further, his style of administration was highly personalised
revealing his strong likes and dislikes. All these qualities made
him easy prey and, in less than three years, many, including
several senior faculty members, turned against him.

In the matter of the appointment of Max Born\index{Born, Max|)} to a
permanent Chair and in inviting other scientists from abroad,
Raman had apparently not taken permission of the management,
the Governing Council of the Indian Institute of Science. Some
of the other charges levelled against him were that he developed
the Physics Department at the expense of other departments, that
he wasted money on beautifying the Institute grounds and
so forth.

While Raman could admittedly be a difficult person, there
is reason to wonder whether it was not just the `scientific politics'
of the times that brought about his demotion. It is a pity that
a scientist of his stature had to suffer indignities which left an
indelible bitterness in him. An institution which is governed by
consensus and which is bureaucratic by nature was totally at odds
with Raman's way of thinking. It is out of this realisation that
he decided to create a research institute of his own, where his
will and wishes would prevail.
\index{Raman, Chandrasekhara Venkata!Traits/Interests|)}

\vskip -.45cm
\heading{The Raman Research Institute}
\addtocontents{toc}{\protect\contentsline{section}{The Raman Research Institute}{\thepage}}
\vskip .1cm

\index{Raman Research Institute|(}

\lhead[{\it\fontsize{9pt}{9pt}\selectfont\thepage}]{\it{\fontsize{9pt}{11pt}\selectfont The Raman Research Institute}}

\noindent
Raman had visions of a private institute in which he
could continue his scientific research after he retired from the
Indian Institute of Science, he had planned for this while still
a Professor at the I.I.Sc., collecting money for such an eventuality.
The Maharajah of Mysore had earlier gifted for this purpose a
lovely piece of land, 11 acres in extent, in one of the prime
localities in Bangalore. Raman arranged to build his institute on
this land and, by the time he retired from the I.I.Sc., in 1948, the
building for his new institute was nearing completion.

The Raman Research Institute is situated on a piece of land
adjacent to that of Kempegowda tower which defines the northern
limit of Bangalore. (The four Kempegowda towers were erected
in the 19th century to mark the limits beyond which the city should
not be extended without ill-luck descending on it, but in recent
years Bangalore has out-grown these limits and has had to contend
with congestion, a lack of water and pollution.) The land on which
Raman built sloped gently towards the north and its soil was
lateritic. The Kempegowda tower stands to the southeast on a
lateritic hillock and the Sun used to light it up in blazing red.
From this tower, you could see the Nandi Hills 30 miles to the
north. To the south lies Bangalore, cradled in a shallow valley.
The adjoining area in those days was occupied by the beautiful
palace gardens and lush agricultural fields. In recent times, the
gardens and open fields have given way to housing developments,
and the area has become heavily built up.

Raman loved the property, every inch of it was sacred to
him. One of the first tasks undertaken by him was to erect a
barbed-wire fencing around it. He then implemented an
afforestation scheme, planting the barren land with flowering
trees, shrubbery, bushes and fast-growing trees for shade and
greenery. The first building that was erected was a two-storey
one of neatly dressed, light grey granite blocks. The building was
essentially designed by Raman with the help of some architects.
Raman believed in large rooms with high ceilings and sumptuous
ventilation. It was a `$\Pi$' shaped building facing the north, the long
side laid out in the east-west direction, with the east and west
wings at the ends. A protruding portico near the western end
served as the main entrance to the building. There were pillared
verandahs all round, both on the ground floor and the first floor.
A large granite staircase faced the entrance. It had a landing
platform at the half-way point, where it turned 180$^{\circ}$ to lead up
to the first floor. These steps were beautifully dressed rectangular
slabs of granite cantilevered from the adjacent wall. Raman used
to point out this feature to visitors and remark that he could take
a full-grown elephant up them to the first floor. The steps had
dressed granite banisters.

On reaching the first floor,\index{Raman, Chandrasekhara Venkata!Traits/Interests} a pillared porch right on top of
the portico invited one's attention. This was a favourite spot of
Raman. From this portico there was a panoramic view of the
northern side, and on a clear day the distant Nandi Hills could
be clearly seen, their well-defined outline resembling that of a
reposing bull. Further details on the hills could be seen with the
help of a telescope. The foreground was somewhat bereft of
vegetation, but the overall picture was of a vista exhilarating to
the mind and a feast to the eye. Raman hardly ever failed to give
a visitor the experience of this sight, his commentary adding
another dimension to the view.

To the east of the building was the majestic tower of the
Indian Institute of Science,\index{Indian Institute of Science|)} with the Institute complex in the
foreground. This sight however did not please Raman. So he
planted an eucalyptus grove to block the view. He never forgot
the unpleasant events connected with his tenure as Director of
the Indian Institute of Science. These had left a deep scar on him.
Sometimes Raman used to make pungent remarks about the
Institute. He, however, would quickly change the subject and
talk about eucalyptus, saying how much he enjoyed the scent
wafted by a favourable wind from that direction.

When Raman shifted his activities in 1949 to the Raman
Research Institute, the building was barely completed. There
was no electricity and the plumbing was just being laid.
In November 1949, I joined Raman as his research assistant.
He appointed \hbox{Padmanabhan}\index{Padmanabhan, J.}\break as his technical assistant about the
same time. A stenotypist was appointed just a little before
us. The three of us were the nuclear staff of the Raman
Research Institute.

Raman recruited his staff in his own way. He was fed up
with rules and regulations followed in big institutions such as the
I.I.Sc. He applied the criteria of ability and proven merit rather
than paper qualifications and evaluated candidates in his own
way. Once, there was a query from the Government as to the
basis on which he recruited his staff. Raman wrote back saying,
``On the basis of proven merit''.

The first technical person Raman appointed was J. Padmanabhan,\index{Padmanabhan, J.} who was recommended to him by
 H. Parameswaran.\index{Parameswaran, H. (H.P. Waran)} The latter had trained Padmanabhan as an optical technologist.
Raman had great regard for Parameswaran who had, at one time,
been Professor at Presidency College,\index{Presidency College} Madras, and had later gone
to Trivandrum to serve as the Director of Industries in Travancore
State. Parameswaran was an expert in optics and had constructed
telescopes and other optical apparatus for research in the subject.
Padmanabhan had, therefore, received excellent training and
turned out to be a master optical technologist. Raman appreciated
very much Padmanabhan's work and the excellence he was able
to achieve. Padmanabhan had the unique distinction of staying
with Raman for twenty years, until the latter's death. He retired
from the Raman Institute in 1984, after 35 years of service.
Padmanabhan served Raman with dedication; Raman, in turn,
liked him very much and treated him very kindly.

In my own case, it was a chance meeting with Raman that
resulted in a most happy and productive association with him
for eleven years, from November 1949 to October 1960. After
obtaining the Bachelor's degree in Science from the University
of Madras,\index{University of Madras} I spent a year-and-a-half in search of a business-related
career. Not finding satisfaction and drawn towards science by
a strong inner urge, I went to the College of Engineering, Guindy,\index{College of Engineering, Guindy}
to work as a research scholar in physical chemistry under a well
known physical chemist who headed the Science Department
there. After a year, I became a demonstrator with some teaching
duties and laboratory work, but spent my major time in research.
My two-year stay at the Engineering College gave me an
opportunity for self-education in electrical and mechanical
engineering, which stood me in good stead later in my career.
Finding that the Engineering College was not an appropriate
place for growth as a scientist, I decided to leave and knock on
the doors of the Indian Institute of Science, Bangalore, but there
was some hitch. However, shortly afterwards, I had to visit
Bangalore, to investigate a business opportunity in non-ferrous
metals. While there, I made a telephone call to Raman and
inquired if I might see him later that afternoon. Luck was on
my side; Raman was in good spirits and said `Yes'. I went to
the Institute and found him taking a walk in the grounds. He
welcomed me and talked in general terms about the Institute he
was planning to create. In between, he asked me what sort of
training I had and then directed some general questions at me.
I distinctly remember one of them concerning clouds. He asked
what clouds were and why they were visible to the eye. I must
have given the right answer, for he said, ``You have some common
sense''. Then he asked me if I knew something about the practical
aspects of electrical engineering. He talked to me about plants,
flowers and about the origin of colours. I had taken a course
in botany and, hence, could respond sensibly.

All this went on for an hour and a half while we took a
leisurely stroll. Then he said, ``I want an assistant to help me
in my research and scientific work, but I have to be careful in
selecting the right person''. I responded, ``Sir if you will give
me a chance I will try my best to serve you''. Without hesitation,
he offered to take me as a Junior Research Assistant, but on one
condition, namely that I would have to be on probation for six
months; if at the end of that period he was satisfied with my
performance, he would confirm me. He jokingly added, ``Maybe
you will turn out to be like a Faraday to Sir Humphrey Davy''.
This was a most exhilarating outcome for me and indeed the
turning point in my life. Nothing like this had ever happened
before or after: a contact with one of the greatest scientists
of the century, which was to set me on a course dearer to my
heart. In retrospect, all the disappointments I had endured were
for the good.

\begin{figure}[p]
\rotatebox{90}{
\begin{tabular}{c}
\includegraphics[scale=1.02]{eps/7.eps}\\
{\fontsize{10pt}{12pt}\selectfont{\em The Raman Research Institute} circa {\em 1953.}}\relax
\end{tabular}}
\end{figure}

This meeting took place on November 1, 1949. Raman
immediately dictated a letter of appointment to his typist,
signed it and handed it to me, asking me to report for work in
ten days' time. He then dropped me at the bus stand. I took the
train to Madras that night and, after spending a week in my
village home left for Bangalore to start work. Padmanabhan\index{Padmanabhan, J.} had
already joined and so had the other staff member, the typist
Balakrishnan.\index{Balakrishnan (typist)} It was from this point that the Raman Institute
began functioning.

Raman, meanwhile, went about earnestly building the
Institute and its facilities for research. He had bought several
microscopes, and there was a roomful of electronic equipment
(U.S. military and air force surplus equipment released by DGTD
to educational and research institutions from what the American
Armed Forces had left behind after World War II). These
included magnetrons, microwave generators, oscilloscopes,
transmitting equipment, servo-systems, ae\-rial cameras, optical
systems, infra-red viewers and detectors. A large number of
machine tools and lathes, and a liquid nitrogen plant, were also
part of this windfall. Radhakrishnan,\index{Radhakrishnan, V.} the younger son of Raman,
was an amateur radio expert and had helped in choosing most
of the electronic equipment. However, most of the items were
not really in working condition, except the optical components
and machine tools. Later, a building was built on the western
side of the main building for the lathes, machines, tools, glass
blowing technology, carpentry and a chemical lab.

Another building called the Spectroscopic Laboratory was
built next to the main building on the eastern side.

When it came to buildings, Raman\index{Raman, Chandrasekhara Venkata!Traits/Interests} loved to have them built
in dressed granite. He also had very definite ideas about their
architecture. The main building of the Institute for instance, was
almost entirely conceived by him; he had thought out exactly how
each room should be utilised. In the main building he provided
space for laboratories, library, museums, offices, reading rooms,
and bathrooms. His reading room on the second floor, adjacent
to the museum hall, was magnificent. It was panelled in teakwood
and had teakwood and glass cases for books. There was corner
shelves on which were arranged several spectacular mineral
specimens. A large teakwood table with a black glass top stood
in the centre of the room, with comfortable cushioned chairs
around it. The Kempegowda tower could be seen through the
eastern windows of this room and the palace gardens from the
southern windows. Altogether this was a lovely room, tastefully
decorated, and with books and journals neatly arranged on the
shelves. Raman used this room to read, write and sometimes
receive visitors. A small adjacent room contained the {\em Memoirs}
of the Institute,\index{Memoirs of the Raman Research Institute@\textit{Memoirs of the Raman Research Institute}} all arranged according to number and accessible
through a door from the reading room. Downstairs, in the north-east 
corner, was Raman's private office. This was close to the
Institute's administrative office where his stenotypists, clerks and
accountants sat. In his private office room, Raman kept his entire
personal collection, including memorabilia, medals, diplomas,
doctoral gowns, diamonds and other crystals. He would answer
telephone calls, dictate letters and transact administrative matters
connected with the Institute from this private office.

Before Raman\index{Raman, Chandrasekhara Venkata!Visits Abroad} retired from the Indian Institute of Science
he visited the United States twice, as a member of the Indian team
to the World Bank. Although the official role he played in these
two delegations is not clear, he made use of the opportunity to
visit several scientific institutions, including the Bell Laboratories\index{Bell Laboratories}
at Murray Hill. More importantly, he acquired an exquisite
collection of mineral and crystal specimens and gem materials
from several dealers in the U.S.A. These had just arrived about
the time we joined the Institute and one of the more interesting
tasks to befall us was to unpack them.

Raman had a grand plan for crystallographic, mineralogical,
gemmological, and geological museums at the Institute and had
reserved several rooms on the first floor for this purpose.
He immersed himself in the task of designing a suitable display
and called in E.K. Govindraj,\index{Govindraj, E.K.} a well-known photographic dealer
and studio owner in Bangalore, to help him. Govindraj, who had
a fine artistic sense, suggested the use of teakwood-framed glass
cases, with sliding doors and inside lighting, to display the
collection. The shelves in the cases were also to be made of plate-glass. The teakwood casings were to be decorated with a design
and fine-polished to bring out the graininess of the wood.
Govindraj engaged carpenters, built the shelves at the Institute
itself and personally supervised their installation. But he sought
Raman's approval at each stage. If Raman did not like anything
that Govindraj proposed, he told him so, and Raman's decision
prevailed. Govindraj, however, was a very diplomatic and sharp
person and he learned very quickly the ways of Raman's thinking.
\index{Raman Research Institute|)}

After the shelves were installed, Raman took great delight
in arranging the specimens himself. He used to carry the specimens
up to the first floor himself and place them on the shelves.
We gave him all possible help. He would have us around him,
then try various positions for a specimen and ask each time,
``How does it look now?'' Thus, considerable time was spent on
arriving at the right setting for each specimen, to ensure that it
caught the light properly and its beauty was displayed in full
measure. These sessions used to last for hours and we greatly
enjoyed being with Raman on these occasions, listening to his
comments. He did not like labels on the specimens and hence
these were removed, but he carried all the information about each
specimen in his head. From him, and because we were with
him all the time, Padmanabhan and myself learnt the details
about most of the specimens, how they were acquired, from which
area and so on.

The most colourful and spectacular crystals were displayed
in a corner room with front and side lighting. The larger mineral
specimens were exhibited with front lighting in a long room called
the Bisseserlal Halwasia museum. There was a long teakwood
case with a glass top and sliding doors in this room and in it were
displayed most of the synthetic gems and crystals. In a small room
adjacent to the corner room, minerals which exhibited
luminescence under ultra-violet (UV) excitation were displayed.
The bright green willemite, the red calcites, blue fluorspar and
the multicoloured franklinite from New Jersey were part of a
spectacular show. They were displayed in a darkened room, but
came to life when UV lamps were turned on. Raman loved to
show the specimens in white light before turning on the UV so
that visitors could see the contrast. It was like being in a fairyland,
where things took shape in splendid colours, as the UV lamps
warmed up. Raman would explain to his audience what
luminescence is and why the specimens showed such colours.
This was one of the main attractions of the museum at the Raman
Research Institute and visitors carried away with them a vivid
impression of the show. In the western wing, one room was set
apart for geological specimens (rocks and rock-forming minerals)
and two rooms for the display of butterflies, beetles, stuffed birds,
iridescent shells and nacre.

Raman equipped the Institute with beautiful museums, 
lecture hall, library, offices and laboratories and carried on his scientific work in it with tremendous enthusiasm and fervor. He also took a few research students, but the Institute was founded primarily for him to work on his interests. 

Raman had aristocratic tastes.\index{Raman, Chandrasekhara Venkata!Traits/Interests} He dreamed of a villa with a sunken garden and a pergola for his residence in the campus of the Raman Research Institute. He again approached the Maharajah and asked him to grant him (Raman) four acres of land adjacent to the southern side of the campus. This was part of the agricultural land. His request was granted and he was
extremely happy about it. Rightaway he fenced the additional
property and planted it with a number of flowering trees and
shrubs. He then built his dream house, a lovely residence, in the
southeast corner of the property. The Director's house was a long
one-storey building in dressed granite, facing a sunken garden
on the southern side and a pergola on the northern side.
The sunken garden was filled with some of the loveliest rose plants
in Bangalore. On the eastern side, Bangalore's northern
Kempegowda tower stood like a sentinel on the red hill. Raman
located his bedroom in such a way that he could see the tower
from his bedroom window. For the pergola he chose a violet and
yellow flowering creeper. A drive-in portico built of granite stood
on this side straddling a circular driveway. On the western side,
Raman planted a variety of flowering shrubs and laid out several
walks. All in all, it was a lovely location and a beautiful home.

Raman did not occupy the house right away and continued
to live in his bungalow in Malleswaram, about three miles from
the Institute. The first occupant of the Director's quarters was
Palmer Craig,\index{Craig, Palmer} a visiting professor of electrical communications
from the U.S.A. to the Indian Institute of Science.\index{Indian Institute of Science} This curious
occurrence happened this way. Craig was looking for a house
when he met Raman casually one day and mentioned his
predicament. Jokingly Raman said, ``I have a lovely house, but
I don't think you will pay the kind of rent I want for it''. Craig
wanted to know how much the rent was and Raman quoted
Rs. 2,000 a month. Craig saw the house and immediately agreed
to pay the rent asked. To an American Professor this was nothing
very extravagant and, in any case, it was a lovely location and
a well-appointed bungalow. Raman was caught in a tight corner,
but agreed to let the house to Craig\index{Craig, Palmer} for two years. Maybe he
thought it would bring in a tidy sum of money which he could
use for the Institute. But towards the end of the contract, he was
getting impatient, for he wanted to move into the house very
badly. Once or twice he remarked to me that he had made a
mistake in letting the house and that he should not have
succumbed to the high rent.

\index{Raman Research Institute|(}
Raman had arranged to build a hostel and two houses in
the campus and wanted me to occupy one of the houses. In fact,
he wanted me to design a compact building with all amenities
and took a great interest in the project. He used to remark, ``You
must live in comfort. Have a nice Western toilet and a geyser
in the bathroom''. He was very considerate to me and gave me
the house free of rent. Living in the campus had advantages as
far as scientific work was concerned, but in those days the Raman
Institute campus seemed so far away from civilisation. The air,
however, was pure and the surroundings lovely. The research
students, my family and the family of Venkatachar,\index{Venkatachar, B.S.} the manager
of the Indian Academy of Sciences,\index{Indian Academy of Sciences} lived in these quarters as one
large family, sharing the troubles and enjoying the amenities.

Before Raman moved into the Director's house at the Raman
Institute campus, he lived in his own palatial house in the
Malleswa\-ram section of Bangalore. This house was known as
{\em Panchavati}\break \hbox{(Hermitage)}, appropriate indeed for his name, and
had extensive grounds with neem, mango, jack and other trees.
It was an old house which Raman had bought for a bargain price.
The story goes that the house had the reputation of being haunted
and hence did not attract buyers. When Raman was considering
buying the house, he heard the story and, apparently, remarked
that he was a greater ghost than the resident one and would soon
drive it out.

Eight miles west of Bangalore, Raman had a lovely estate
of about 100 acres near a village called Kengeri. The estate had
a nice bungalow in it and consisted of agricultural land as well
as extensive groves of trees. A stream ran along its southwestern
perimeter. \hbox{Raman} used this property mainly for weekend
recreation. He loved to walk there. Usually he left the Institute
on Saturday afternoons and reached the estate before sunset.
He would then walk to a particular point in the estate and watch
the sunset. Lady Raman\index{Raman, Chandrasekhara Venkata!Raman, Lokasundari} accompanied him on these trips to
Kengeri and attended to his needs. The bungalow was fully
equipped and furnished. There were workers to take care of the
agricultural operations and maintain the property. Lady Raman
managed the servants and raised vegetables and other crops.
On Sunday mornings, Raman would take a long walk, enjoying
the scenery. After lunch and some rest, he and Lady Raman
would return to Bangalore in the late afternoon. Raman wanted
to locate a centre for astronomical research in this estate.

In the late Fifties, Raman got about five acres of land
adjacent to the Institute property on its eastern side. The City
Improvement Trust Board of the Government of Karnataka,
which had by then taken over most of the land belonging to the
Maharajah, allotted this land to the Institute --- but at a price;
Raman paid about Rs. 3,00,000 for it. This land was part of the
lateritic hill sloping towards the main road, on the northern side
of the Kempegowda tower.

Raman had ideas of putting up additional buildings on this
property as part of his expansion plans for the Institute.
He actually arranged to have the foundations laid for a building
to house a new library. Raman thought that the City Improvement
Trust would claim back the land if it was left unbuilt. In the
Seventies, long after his death, several new buildings were indeed
put up on this land, including a building which houses an
impressive radio telescope for millimetre wave radio astronomy.
Raman's foresight and vision have proved absolutely correct.
The value of the land has multiplied by a factor of two or three
hundred at current prices and had Raman not acquired this land,
it would have been impossible for the Institute to expand one
of its principal activities in such a convenient location.

Raman also acquired a few acres of land in Madras, in a
locality which became one of the nicest residential sections in the
city. He put up a house there and wanted to establish a branch
of the Raman Institute in Madras devoted to mathematical
sciences. This property was, however, sold by him in the Sixties
for over a million rupees.

\medskip
\noindent
{\em Early Days}
\smallskip

For nearly two years after my joining, that is until the end
of 1951, the Raman Research Institute did not have any electricity.
We set up a dark room for photography, and darkened the end
room in the west wing to carry out optical studies with sunlight.
A pillar with a platform was built at a suitable distance outside
this room on the southern side and sunlight was reflected with
a hand-operated heliostat. A heliostat is a device used to reflect
a beam of sunlight at a constant angle and consists of a mirror
mounted on an axis which points at the Pole Star. It follows the
sun and throws the reflected light beam on the same spot.
The device is usually driven by a clock mechanism, but in the
early days of the Institute we used the human hand to operate
our heliostat. One of the laboratory attendants was stationed
outside and he rotated the mirror axis as smoothly as possible
to keep in step with the sun's motion. This was a tedious job
and, at times, the beam disappeared when the attendant dozed
off. At other times, the beam was kept in order with a knock
on the window to give the necessary feedback. Sometimes there
was a constant struggle between the experimenter and the
attendant. Amazingly, it all worked satisfactorily and we got a
lot of research accomplished this way, despite the handicap of
not having electricity.

\eject

Raman was a great believer in the efficacy of sunlight for
light-scattering experiments, for it was this technique that
provided him the first clues to the Raman scattering process.
Therefore, the lack of electricity did not deter Raman from
carrying out first-class experiments. Some very fine studies were
performed on the optical phenomena exhibited by iridescent
feldspars, {\em viz} labradorite, moonstones and opals. Filtered sunlight,
using a Wood's glass filter, proved to be excellent for exciting
the fluorescence of diamond, and in particular to photograph
the luminescence patterns exhibited by diamond cleavage plates.
We used to sit for hours in a darken€d room, with a beam of
sunlight, to probe the optical phenomena exhibited by gems, and
come out with illuminating findings.

\vskip .06cm

Whether it was the luminescence of diamonds, the multicoloured spectrum of opal, the labradorescence of feldspars,
or the Schiller effect in moonstones, Raman's wonder and
amazement knew no boun\-ds. Raman thought aloud when making
observations and it was an unforgettable experience to listen to
him. The scene at such times would be something like this:

\vskip .06cm

He would be looking at a sample with the sunlight shining
on it and remark, ``I say, you won't believe what I see. It is a
beautiful effect''. Then, after a while, he would say, ``I think
1 see it, but you know it comes and goes''. The standard response
of the bystander would be ``Yes, sir''. His next statement, after
a few more observations, would be, ``I don't think I was right
in thinking that way. Now I don't see it. I think I was very foolish
to have mistaken that spurious effect as the real one''. It would
then be awkward for the bystander to say ``Yes'', in case Raman
was offended by the confirming ``Yes''. However, he need not
have worried because Raman was only thinking aloud and usually
what a bystander said was irrelevant to him at such times. He only
wanted the onlooker to share his own excitement.


In 1950, Raman recruited seven research scholars,
T.K. Srinivasan,\index{Srinivasan, T.K.} a geologist with an M.Sc. in geology from
Mysore University, K.S. Viswanathan\index{Viswanathan, K.S.} with a Master's degree in
mathematics from Madras University, D. Krishnamurti\index{Krishnamurti, D.} with a
B.Sc. (Hons.) degree in physics, also from Madras University,
S. Chandrasekhar\index{Chandrasekhar, S.} with an M.Sc. in physics from Nagpur,
A.K. Ramdas\index{Ramdas, A.K.} and M.R. Bhat\index{Bhat, M.R.} with B.Sc. (Hons.) in physics from
Poona University and S. Venkateswa\-ran\index{Venkateswaran, S.} with a B.Sc. (Hons.)
physics from Madras University and a professional certificate in
communication engineering. S. Pancharatnam\index{Pancharatnam, S.} joined in 1954;
he had taken an M.Sc. in physics from Nagpur. The Council of
Scientific and Industrial Research also granted several senior and
junior research scholarships and some research scholars were
appointed on these.


Raman wanted a geologist because of his interest in the
physics of minerals. Srinivasan was sent on several expeditions
to collect rocks and minerals and, as a result, the geological
collection at the Institute grew rapidly. Raman asked Srinivasan
to work on the optical properties of minerals and then on the
magnetic properties of rocks, but Srinivasan's studies never
took off the ground. Somewhat disheartened, he left the Institute
after three years to take up a job with the Associated Cement
Company as a geologist. Venkateswaran\index{Venkateswaran, S.} also left after a
brief stay.

In general, the scholars who were experimentally inclined
became very dispirited because there was no electricity for two
years and they could not carry out any experimental studies.
Those who took to theoretical studies, like Viswanathan\index{Viswanathan, K.S.} and
Chandrasekhar,\index{Chandrasekhar, S.}\break were active in research. Because of this situation,
the research scholars were asked to attend the theoretical physics
lectures given by B.S. Madhava Rao\index{Madhava Rao, B.S.} and K. Subbaramiah\index{Subbaramiah, K.} at
Central College, Bangalore.\index{Central College, Bangalore} Krishnamurti\index{Krishnamurti, D.} was assigned to a
theoretical problem, to calculate the elastic constants of diamond
from its vibrational spectrum, and he made a success of it.
Chandrasekhar took to theoretical formulation of the optical
activity in crystals, while Viswanathan went into lattice dynamical
theories. Although the lecture programmes gave a good theoretical
grounding, lack of experimental contact had a deleterious effect
on the experimentally-inclined scholars. The only experimental
research programme was the personal programme of \hbox{Raman} in
which I had an active role assisting him. Srinivasan\index{Srinivasan, T.K.} joined in one
project, namely the study of moonstones, and subsequently co-authored one of the papers.

Towards the end of 1951 the electrification of the Institute
was completed and I had a major role to play in it. The day
electricity was switched on from the main power line there was
jubilation among everyone. Raman was particularly pleased.
One of the first things he did was to run upstairs, switch on the
ultra-violet lamp in the luminescent mineral room and enjoy the
spectacular display of the fluorescent colours. By the end of 1952
we had spectrometers, X-ray units, and a workshop with full-fledged mechanics. I set up most of this equipment and X-ray
diffraction studies were now entrusted to me completely.

As various facilities were added, Raman recruited technicians, 
machinists, carpenters and a librarian to man them. He was
very particular to recruit a book-binder to ensure that journals
received in the Library were bound attractively in leather, with
their titles printed in gold. Most of the technical people he
recruited stayed on at the Institute and made it their career.
Among them, a young glass-blower by name Balakrishnan,\index{Balakrishnan (glass blower)} who
joined the Institute around 1952, was particularly well liked by
Raman, for he turned out to be a person with all-round skills.
He was an expert glass-blower, but could also operate a
wide variety of electrical and electronic instruments. Raman
valued ability and skill and spontaneously appreciated good and
efficient work.

\index{Raman, Chandrasekhara Venkata!Traits/Interests|(}
Raman was very prompt in his correspondence, being quite
busi\-ness-like. If a letter needed reply, he would send back one
right away. Decisions were taken by him quickly and he was easily
accessible to everyone. There were no formalities and you did
not have to go through a secretary. Even the gardener could go
in and talk to him. In fact, the gardener received special treatment.
Any matter connected with trees or shrubs received Raman's
immediate and personal attention.

Raman had a driver by name Parthasarathy\index{Parthasarathy (driver)} who had been
with him for a long time. He was quite a character, but was a
good and careful driver. Raman had absolute trust in
Partha\-sarathy and would believe whatever he had to say about
the car, its health and its performance. Parthasarathy was also
Raman's time-keeper. Raman had many wrist-watches, but he
either forgot to wear them, or even if he had one on his wrist
it would not be showing the proper time, for he would not have
wound it. Driver Parthasarathy had an old wrist watch, perhaps
a quarter-century-old, with only the hour-hand on it. Raman
always asked him what the time was before he started on an
outing. And Parthasarathy would quite correctly estimate the time
from the position of the hour hand and announce it. Then the
command would come from Raman, ``Let us go. It is getting
late''. Raman had an old Willys sedan of light grey colour which
gave excellent service. In 1951, or thereabouts, he bought a new
Studebaker, a large car two-tone in colour, grey and green.

Raman was an early riser and was always ready for work
at 6 a.m. or even earlier. Some days he used to walk the two-and-a-half miles
to the Institute early in the morning, cutting across Malleswaram
and taking the short cut through Sankey Tank Road. Lady Raman\index{Raman, Chandrasekhara Venkata!Raman, Lokasundari}
would later send the car with his breakfast, usually a piece of
toast, banana and coffee. One day, Raman told us that he would
like to become independent of the driver. So he bought a new
bicycle. For two days, Raman, Padmanabhan\index{Padmanabhan, J.} and myself used
to ride our bicycles to the Institute. This, however, did not work
out, for Raman became very tired pedalling the bicycle up the
gradient, all the way from the Institute of Science Circle to
Hebbal. Further, Raman did not pay much attention to traffic
on the road and often strayed to the wrong side. This proved
dangerous and Lady Raman forbade him from cycling to
the Institute. He then gave his cycle to Padmanabhan. It was
strange the way Raman would get on his bicycle. He would put
one foot on the fork and climb on, take a few hops with the
other foot and then settle on the seat. I presume that is the old
way of doing it, but it was amusing to watch a 61-year-old
Nobel Prize-winning scientist hopping on to his bicycle seat in
this fashion, before riding alongside two of his assistants. Raman,
however, did not usually pay much attention to what others
might say.

\newpage

There was another occasion when he hurt his toe and could
not wear shoes. For nearly a month he went about his business
barefooted. He would of course be fully dressed, but there would
be no shoes on his feet! It was funny to watch Raman walking
barefoot, unmindful of what others might think of him.

On the first of every month it was almost a religious duty
with him to go to the Central Bank of India in the city and return
with crisp new currency notes for the disbursement of salaries.
If he was ill and could not go, either Lady Raman went or
he would ask me to go. He loved to see the salaries paid promptly
on the first of every month and the employees kept happy
and satisfied.

Raman instantly appreciated a job well done, whether it was
scientific research, technical work or any other kind of activity.
Padmanabhan used to make glass and quartz spheres of various
sizes for Raman and had evolved a clever technique to make them.
They came out beautifully. There is a magical quality to a large
sphere of quartz; they are fascinating to look at. Raman used
to go into raptures whenever Padmanabhan finished a sphere and
placed it in his hands. He would look at it and say, ``Oh, is it
not lovely? Padmanabhan, this is beautiful! This is fantastic!''
He would place the quartz sphere between crossed-polarisers and
admire the concentric coloured rings that appeared when viewed
along the C-axis of the quartz crystal.

Balakrishnan\index{Balakrishnan (glass blower)} had a lovely hand and Raman often asked him
to draw on the blackboard, using coloured chalk, pictures and
equations for his lectures. In those days, there were no vugraph
machines, but Raman invented his own vugraph by having the
relevant things drawn beautifully on the board. Raman
appreciated Balakrishnan's artistic work on the blackboard very
much and used to shower praise on him. Such was his nature.
\index{Raman, Chandrasekhara Venkata!Traits/Interests|)}

\medskip
\heading{The colours of minerals, gemstones and crystals}
\addtocontents{toc}{\protect\contentsline{section}{The colours of minerals, gemstones and crystals}{\thepage}}
\smallskip

\lhead[{\it\fontsize{9pt}{9pt}\selectfont\thepage}]{\it{\fontsize{9pt}{11pt}\selectfont The colours of minerals, gemstones and crystals}}

\noindent
Raman was deeply interested in the colours exhibited by
rocks, minerals and crystals. He carried out several important
investigations on them, in which some of us collaborated. He was
proud of the fact that his museum collection served as the source
material for this research. He told visitors, ``I collect these items
not just for display. They are the source material for my
research''. He had a lovely collection of what are known as
iridescent feldspars, which are naturally-occurring silicate minerals
that exhibit very colourful optical effects. Raman was deeply
attracted by these feldspars known in mineralogical literature as
labradorite, peresterite, murchisonite, amazonite, moonstone
and sunstone.

While the labradorite and peresterite specimens were large,
with flat, polished surfaces, the moonstone, murchisonite and
sunstone were cut and polished as cabochons or hemispheres.
There were also uncut crystals of these in the collection, with
crystallographic cleavages. Among these, the iridescence of
labradorite is most spectacular; it is in the nature of a brilliantly-coloured metallic reflection coming from the depth of the crystal
when the specimen is held properly, with respect to the incident-light, and viewed. Further, the dark background of the labradorite
accentuates the colours, which change hue when the angle of
observation and incidence of light are changed by tilting the
specimen. A properly sectioned and polished specimen of
labradorite is gorgeous to look at and immediately arrests the
viewer's attention.

The optical phenomenon exhibited by moonstone is in the
nature of a diffuse reflection and the colour varies from a deep
blue to bluish-white to a silvery-white sheen. Most of the gem-quality moonstones
 originate from Ceylon (Sri Lanka) or Korea
and the blue diffusion observed in them is sky blue in colour.
There were some very fine specimens of Ceylon and Korean
moonstones in Raman's collection.

One of the very first research projects undertaken by Raman
at the Raman Research Institute\index{Raman Research Institute|)} was the study of `iridescent
feldspars', in which I assisted him. As remarked earlier, there
was no electricity in the Institute at the time, so Raman used
sunlight to study the optical phenomena. In fact, this proved to
be the best way, for, a narrow beam of sunlight incident upon
the surface of a sample immersed in a suitable fluid medium
revealed rich details of the optical effect exhibited by the stone.
Thus, if the optical effect was in the nature of a reflection,
coloured reflections were observed on a white card held at the
appropriate distance from the sample surface. In the case of
moonstone, a diffuse reflection was observed spreading out into
an elliptic halo, blue or whitish in colour, depending on the
moonstone. In the case of sunstone, a golden-coloured halo
appeared on the white card.

\vskip .05cm

All these details gave clues about the optical effect in question
and, with Raman's intuitive understanding of optical phenomena,
scattering of light and diffraction effects, the underlying cause
of the observed effects crystallised. For us it was the greatest
opportunity to learn physics, sitting with Raman during an
experimental observation session. We obtained an intuitive
understanding of optical principles just listening to his comments
during these experiments. Such fast understanding could not have
come from searching a dozen books.

\vskip .05cm

From all these studies it became clear that the optical effects
exhibited by iridescent feldspars in general originated from the
optical heterogenities in the mineral. The size and shape of the
heterogenity determined the character of the optical effect. Thus,
in labradorite, the brilliant iridescent colours were due to the
presence of one of the component feldspars in the form of thin
lamellae inside the specimen, distributed more or less uniformly
over large areas. Because of the differences in the refractive
properties between the lamellae and the matrix feldspar, the light
was diffracted or reflected from inside the matrix crystal, the
colours depending on the thickness of the lamellae, the angle of
incidence and the reflection of the light. Detailed optical studies
on moonstones revealed that the blue or silvery diffuse reflections
were because of the presence of the finely segregated component
feldspar (soda feldspar) with dimensions close to the wavelength
of light inside the moonstone. Further, these segregations, or the
optical heterogeneity, had a cigar-like shape, all arranged more or
less in parallel orientation in the matrix crystal of potash feldspar.

\eject

Raman got interested in opal, a well-known, naturally-occur\-ring 
form of silica. The so-called precious opals are highly valued
as a gemstone. He had a fine collection of opals originating from
Australia, Hungary and Mexico. Some of them were beautiful
specimens. A very detailed study of the optical characteristics
of the reflections were made along with some X-ray structural
studies. From these, the conclusion was drawn that opaline
material was basically amorphous and possessed two different
types of local structure, namely cristobalite-like and quartz-like.
The studies suggested that the difference in refractive index gave
rise to the optical heterogenity. However, a more recent electron
microscopic study of the reflecting lamellae in opals has shown
that there is a periodic structure in them, with regularly spaced
air inclusions in the opaline matter, which are believed to be the
cause of the bright-coloured reflections emanating from the
stones. Opals have recently been made synthetically and these
are known as `Slochum stones'.

\vskip .05cm

Raman conducted extensive optical studies on pearls,
ame\-thysts, iridescent agate, jade and several other forms of silica.
In each case, it was the colour, or the optical effect that attracted
his attention and he had novel explanations for every effect.
Two other systems in which Raman got very interested were
iridescent potassium chlorate and calcite. Krishnamurti\index{Krishnamurti, D.} was involved
in the study of iridescent potassium chlorate and obtained some
beautiful spectra of the monochromatic reflections from
potassium chlorate crystals. The iridescence of potassium chlorate
is an optical effect arising out of multiple twinning of the crystal
on a fine scale, repeated with amazing precision. When white light
is incident on such a twinned crystal, brilliant monochromatic
reflections are observed. The reflection is analogous to a Bragg
reflection of an X-ray beam by a crystal lattice. In the case of
potassium chlorate, the spacings are of the order of a wavelength
of light and, hence, give rise to Bragg-like reflection of the light.
Krishnamurti studied this phenomenon in great detail. Raman
and Krishnamurti published a series of perceptive papers
elucidating the origin of the effect.

\vskip .05cm

Another beautiful optical study, in which Ramdas collaborated 
with Raman, concerned iridescence in calcite. In calcite,
the iridescent colour was again shown to be due to twinning, but
in this case even a simple twinning gave rise to a bright-coloured
reflection at the twin boundary. This was shown to be due to
the large magnitude of the difference in refraction and dispersion
associated with the twin members.

All these studies were published in the {\em Proceedings of the
Indian Academy of Sciences}\index{Proceedings of the Indian Academy of Sciences@\textit{Proceedings of the Indian Academy of Sciences}} and Raman had them reprinted
under the title {\em Memoirs of the Raman Research Institute}.\index{Memoirs of the Raman Research Institute@\textit{Memoirs of the Raman Research Institute}} Raman
spent considerable time in writing his papers and paid great 
attention to English, grammar and style. He often used Latin
terms in his papers to make a point more emphatic. Papers written
by Raman were like a piece of literature, a delight to read.

Raman immensely enjoyed his investigations on gems and
minerals. This body of work reflected Raman's taste for aesthetics
in physics. He loved to demonstrate his findings to visitors and
he would explain to them in simple language some of the more
esoteric optical phenomena. Maharajahas, prime ministers,
politicians, officials, students and laymen all came to visit the
Institute and see Raman. They went away enthralled as much
by his vast collection of gems and minerals as his inspiring tales.

\bigskip
\index{Raman, Chandrasekhara Venkata!Traits/Interests|(}
\heading{Love of diamonds}
\addtocontents{toc}{\protect\contentsline{section}{Love of diamonds}{\thepage}}
\smallskip

\lhead[{\it\fontsize{9pt}{9pt}\selectfont\thepage}]{\it{\fontsize{9pt}{11pt}\selectfont Love of diamonds}}

\noindent
Raman's love of diamonds is well-known. Describing
diamond as the ``king of solids'', he carried out extensive studies
on it. The first light-scattering measurement on diamond
was initiated by Raman and he assigned the task to his brother,
C. Ramaswamy.\index{Ramaswamy, C.} The story goes that when Ramaswamy
got married, just prior to joining \hbox{Raman} in Calcutta to work with
him, his father-in-law had presented him with a diamond ring,
as was customary in those days. When Raman noticed the stone,
he was apparently fascinated by its sparkle and clarity and
suggested that Ramaswamy record the spectrum of the scattered
light from that very diamond. In this study, the strong vibrational
line characteristic of the diamond lattice was recorded for the
first time. It was also discovered that the diamond exhibited a
blue fluorescence when excited with ultra-violet light. A scientific
paper was written on this phenomenon and, thus, diamond
entered Raman's life.

\vskip .06cm

Raman had a collection of about 600 diamonds of different
kinds and origins. These were from various sources, sometimes
received as gifts and sometimes bought. He used them all in his
studies, classified them according to their properties and boxed
them beautifully. He gave special names to them and had a fund
of stories about each.

\vskip .06cm

Raman was decorated by the Maharajah of Mysore with a
pendant studded with 63 diamonds at the time (1935) he was
honoured with the title {\em Rajasabhabhushana}. This beautiful piece
of jewellery promptly entered scientific literature, for Raman
studied the luminescence characteristics of the diamonds set in
the pendant and drew several interesting conclusions about the
nature and intensity of luminescence exhibited by diamonds when
excited by ultra-violet light. Raman and his collaborators took
to the study of diamonds intensively in the Forties they published
a series of articles on the subject in the {\em Proceedings of the Indian Academy of Sciences}.\index{Proceedings of the Indian Academy of Sciences@\textit{Proceedings of the Indian Academy of Sciences}}

\vskip .06cm

The diamond collection was kept under lock and key, being\break
shown only to chosen visitors. A fortunate few saw the splendid
luminescence of some of his brilliant fluorescing diamonds, a
phenomenon best seen with a beam of condensed and filtered
sunlight in a darkened room. The filter was made of Wood's glass,
which excluded the visible rays and admitted only the ultra-violet.
The tabular diamond Raman had christened as ``my Kohinoor''
exhibited a brilliant blue fluorescence, almost lighting up the
room when placed in the filtered and condensed beam of sunlight.
Another gem-cut diamond named `green diamond' shone with a
dazzling green colour when thus illuminated. There were other
stones in the collection as well, whose luminescence varied from
bluish white to red, with intermediate colours along the line. When
these diamonds came to life with their brilliant fluorescence,
Raman's eyes would light up and his expression ``Isn't it
beautiful?'' was contagious. Anyone who saw this display could
never forget either the diamonds or Raman's way of communicating the aesthetic experience.

\vskip .06cm

In the early days at the Raman Institute,\index{Raman Research Institute} Raman used to
impress visitors by showing them the luminescence in diamond
and explaining to them how diamonds should be tested before
buying. The so called `blue jaggers' were the favoured type in
the Indian jewellery market. In strong white light, a `blue jagger'
shows a light bluishness in its body, apart from the usual fire
and brilliance of the diamond. This blueness actually comes from
the blue luminescence of the diamond. A `blue jagger' under ultraviolet 
light exhibits a strong blue luminescence; this is a simple
test to make. Further, diamonds can have minute defects which
are only observable under a magnifying lens or a microscope.
There can be black spots due to graphitic carbon, or fine cracks.
The presence of these defects reduces the value of the diamond.
Similarly, the body colour can vary from pure white to yellow
to green to brown, and black in the worst case.

\vskip .06cm

Anyone buying a diamond should watch for the three C's
as the trade calls it: colour, clarity and cut. A gem diamond should
be examined at least with a 10x lens for black spots and cracks
inside, and for fracture and quality of the cut outside. With the
proliferation of simulated diamonds, a very quick test for
diamond would be to take the Raman spectrum with a laser.
A real diamond shows a characteristic Raman line at 1332 cm$^{-1}$ 
displaced from the laser line. Such a test can be made in a matter
of minutes with modern instrumentation.

\vskip .06cm

Now and then, someone would come to the Raman Institute\index{Raman Research Institute}
carrying a bag with a stone in it, believing it was diamond. Raman
could tell by inspection whether it was diamond or not, but he
always tested for luminescence, to be certain. Sometimes he would
test the material for hardness and birefringence, the latter with
crossed polarisers. The stone invariably would turn out to be a
lump of quartz at best, to the utter diappointment of the seeker
after riches. Others used to seek Raman's advice before buying
diamonds for earrings, pendants or other jewellery. In the early
days, Raman would amuse himself by looking at them if he was
in a good mood, and the visitor was known to him or Lady
Raman. In later days, he used to delegate this task to me or to
Padmanabhan.\index{Padmanabhan, J.} After I left the Institute, Padmanabhan turned
his expertise into a small but lucrative side-business.

\vskip .06cm

Raman's collection also included a large number of plates
which are called `mackels' in the diamond trade. `Mackels' can
be bought in sizes ranging from a few millimetres in linear
dimensions, and 1 to 2 mm in thickness. These are natural
cleavage plates. Although diamond is a very hard substance, in
fact the hardest substance known to science, it can be cleaved
very easily with a sharp knife-edge by the experienced cutter.
To achieve this, the diamond has to be held suitably, with the
knife-edge positioned along the proper crystallographic direction,
usually along the so-called octahedral plane. If everything is right,
a sharp knock on the knife will split the diamond perfectly along
the desired direction. Diamond is so hard that only diamond can
be used to grind or polish diamond. Diamond-cutting, shaping,
grinding and polishing is an old art, but modern technology has
brought some innovations into the trade.

\vskip .06cm

These cleavage plates were very useful for certain types of
study, particularly for observing the perfection and the state of
strain in them. The majority of cleavage plates exhibit
luminescence patterns when illuminated by UV light, which is
to say that the luminescent areas are not uniform but showed
geometric patterns. Filtered sunlight is one of the most effective
ways of revealing these patterns and for photographing them.
I have photographed scores of these luminescence patterns
for Raman. These plates also exhibit what is known as
birefringence patterns when examined between crossed-polarisers.
These patterns bear a strong resemblance to the luminescence
patterns. Evidently the two have a common origin and Raman
held the view that they were connected with the fundamental
structural properties of the diamond.

\vskip .06cm

Raman was well aware that India was the original home of
some of the best-known diamonds in the world, several of which
had found their way to other parts of the world from India and
helped to spread the fame of this gemstone. He had in his
collection a number of diamonds collected from Panna in what
is now the state of Madhya Pradesh. Writing about the Panna
diamonds in his collection, he said:
\begin{quote}
{\fontsize{10}{12}\selectfont
``At the present time, diamonds are still forthcoming at and near
Panna in Central India. Two visits to Panna made by the author
many years ago gave him the opportunity of inspecting numerous
specimens of the diamonds found in that area in their original
form as crystals. Particular mention should be made of the
magnificent set of 52 uncut diamonds ranging in size from
25 carats down to 2 carats strung together into a necklace by a
predecessor of the Maharajah of Panna. The beauty of the
necklace arises from the lustre and shapeliness of the gemstones.
The two visits to Panna also enabled the author to acquire for
his collection a set of specimens for a more detailed study.

The external features exhibited by the Panna diamonds in
the collection are highly instructive. They are of very varied sizes
and shapes. Two of them present a close resemblance to octahedra
in their general shape. But the octahedral edges are not seen and
indeed there is not the slightest indication of their presence. On the
other hand, the edges along which the diagonal planes of
symmetry meet the curved surface of the diamond are conspicuously visible. 
The six points where these planes intersect in pairs
are located at the six vertices of the pseudo-octahedral form of
the crystal, while the eight points at each of which three planes
intersect appear at the centres of its eight curved faces.
These features indicate that while the diamond has the inner
symmetry of the tetrahedral class, its external form which mimics
octahedral symmetry is the result of the interpenetration of
oppositely directed tetrahedral forms within the diamond.

In some of the Panna diamonds, the lower or tetrahedral
symmetry is much more obviously exhibited in the externally
observed forms of the diamond than in others. There are also
several specimens in which the external shape of the diamond
resembles neither an octahedron nor a tetrahedron but is almost
spherical in form. But in all cases the curved surfaces clearly
exhibit the ridges where they are intersected by the six diagonal
planes of symmetry of the tetrahedron.''}\relax
\end{quote}

\newpage

\heading{The physics of diamond}
\addtocontents{toc}{\protect\contentsline{section}{The physics of diamond}{\thepage}}
\smallskip

\index{Raman, Chandrasekhara Venkata!Papers/Publications/Addresses|(}

\lhead[{\it\fontsize{9pt}{9pt}\selectfont\thepage}]{\it{\fontsize{9pt}{11pt}\selectfont The physics of diamond}}

\noindent
Raman had a passionate interest in the physics of diamond
and came back to the subject several times during his lifetime.
First he was interested in understanding its vibrational spectrum,
which led him to formulate his theory of lattice dynamics. Then
he proposed that there existed two forms of diamonds, a theory
he based on certain characteristic properties diamonds showed.
From a study of the diffraction of X-rays by diamond, Raman
proposed a new kind of X-ray reflection which he chose to call
``quantum X-ray reflections''. He filled the {\em Proceedings of the
Indian Academy of Sciences}\index{Proceedings of the Indian Academy of Sciences@\textit{Proceedings of the Indian Academy of Sciences}} with his memoirs.

According to Raman, carbon atoms, which have s-p
bonding, form a crystalline solid in diamond with tetrahedral
symmetry (Td). This form, lacking in centre of inversion, is ultra-violet- and infra-red-absorbing, blue luminescent etc. The other
form of diamond has full octahedral symmetry (Oh) and is ultra-violet- and infra-red-transparent and is usually nonluminescent.
Raman described the birefringence and luminescence patterns,
X-ray topographs and other properties of diamond in terms of
these two forms of diamonds. He proposed that these properties
owed their origin to the interpenetration of the two forms.

Although these ideas appeared to be consistent, some crucial
experiments around 1958 established that the difference in the
physical properties of diamond were correlated with the presence
of nitrogen as a substitutional impurity in diamonds and had
nothing to do with its fundamental character. Diamond
crystallises only in one form, namely (Oh) symmetry, having the
full symmetry of the cubic system. Further, the non-Bragg X-ray
reflections exhibited by certain diamonds, which Raman believed
originated from the interaction of the lattice vibration with
the Bragg planes, were precisely those with nitrogen impurities,
the nitrogens having formed platelets, which gave rise to
non-Bragg spots.
\index{Raman, Chandrasekhara Venkata!Traits/Interests|)}

\medskip
\heading{`The Physiology of Vision'}
\addtocontents{toc}{\protect\contentsline{section}{`The Physiology of Vision'}{\thepage}}
\smallskip

\lhead[{\it\fontsize{9pt}{9pt}\selectfont\thepage}]{\it{\fontsize{9pt}{11pt}\selectfont `The Physiology of Vision'}}

\noindent
In the Sixties, Raman got very interested in the physiology
of vision. He educated himself very thoroughly on the anatomy
and \hbox{physiology} of the eye and how it functioned as the visual
apparatus {\em par excellence}. He used to talk to visitors about rod
vision, cone vision, colour blindness and acquity of vision.
He carried out very simple experiments with colour filters, using
himself and others as guinea pigs. The culmination of this work
was the publication of a treatise entitled {\em The Physiology of Vision}.

Raman came to the conclusion, as stated in his own words,
that there is no reason to believe that the ideas regarding the nature
of vision and of visual processes inherited from the 19th century
would be sustainable at the present time, either on theoretical
grounds or even as purely empirical descriptions or interpretations
of the observed phenomena. This conclusion was based on many
observations he made in regard to matters like the colours
exhibited by natural objects such as the night-sky, the foliage and
flowers, the birds and butterflies and so on. His book has the
stated purpose of setting out in a systematic manner the
procedures he adopted and the results he obtained in carrying
out several experiments relating to vision and colour. He wrote
that his was an independent study without being influenced by
ideas and beliefs inherited from the past. He used to remark, half
jokingly and half seriously, ``I may get another Nobel Prize for
this work''. The topic was indeed very interesting and timely and,
in fact, a Nobel Prize was shared by Ragnar Granit,\index{Granit, Ragnar} Keffer
Hartline\index{Keffer, Hartline} and George Wald\index{Wald, George} in 1967 for their discoveries concerning
the primary physiological and chemical visual processes in the eye.

It is interesting to note that the topics dealt with in the book
are as diverse as flowers, gemstones and the structure and
functioning of the retina. While writing about flowers, Raman
talks of asters, orchids, roses and so on. While writing about
gemstones, he talks of emeralds, the red rubies of Burma and
the blue sapphires of Ceylon. While writing about the retina, he
describes his own technique and methods of observation which
enable us to view the living retina and thereby gain some
understanding of its structure and functioning. These are topics
which unquestionably cover a very wide range and give us some
insight into Raman's interests and into his thinking in the last
years of his life. In fact, during the years when he was working
on problems of vision, he used to lead many of his visitors into
a darkened room, telling them with enthusiasm that he would
let each one see his own retina, by adopting a novel procedure
discovered by Raman himself. The following is a description of
this procedure:
\begin{quote}
{\fontsize{10}{12}\selectfont
``The technique employed is the use of a colour filter which
freely transmits light over the entire range of the visible spectrum
except over a limited and well-defined region which it completely
absorbs. With suitable dye-stuffs in appropriate concentrations
it is possible to prepare colour filters of gelatine films on glass,
exhibiting the spectroscopic behaviour described. Holding such
a colour filter before his eye, the observer views a brilliantly
illuminated screen for a brief interval of time and then suddenly
removes the filter while continuing to view the screen with his
attention fixed at a particular point on it. He then observes on
the screen a picture in colours, which is the chromatic response
of the retina to the light of the colour previously absorbed by the
filter and which impinges on it when the filter is removed.
Actually, as will become clearer presently, what the observer sees
is a highly enlarged view of his own retina projected on the screen
and displaying the response of the retina in its different areas
produced by the incidence of the light of the selected wavelengths.
By using a whole series of colour filters whose characteristic
absorptions range from one end of the visible spectrum to the
other, we are enabled to explore the behaviour of the retina over
an extensive region under excitation by light of different wave-lengths
 which in the aggregate cover the entire visible spectrum.''}\relax
\end{quote}

His observations on a few other topics dealt with in this
treatise, such as the dyes used in textiles and night-blindness, are
quite original. In fact, the treatise is a remarkable piece of work
in that it presents a new approach to the whole subject of the
physiology of vision.
\index{Raman, Chandrasekhara Venkata!Papers/Publications/Addresses|)}

\newpage

\heading{Butterflies}
\addtocontents{toc}{\protect\contentsline{section}{Butterflies}{\thepage}}
\medskip

\index{Raman, Chandrasekhara Venkata!Traits/Interests}

\lhead[{\it\fontsize{9pt}{9pt}\selectfont\thepage}]{\it{\fontsize{9pt}{11pt}\selectfont Butterflies}}

\noindent
The museum of the Raman Research Institute contains a
beautiful collection of beetles and butterflies. Raman was
fascinated with the colours of these insects, the most colourful
of which were the butterflies called Morpho Brazilius and certain
Himalayan varieties. These large-winged butterflies, exhibiting
a beautiful blue iridescent colour, are spectacular when their wings
are spread out.

The origin of the colour in beetles and butterflies interested
Raman and he wrote a paper on the subject, showing that
a regular periodic structure in their wings produced the beautiful
colours, due to the diffraction of light. The colour depended upon
the angle of observation and was a brilliant metallic blue or
bluish-green.

In the world of butterflies, the wings getting their colour due
to the absorption of light by pigments is more common. Pigments
selectively absorb certain colours from the white light and this
endows the specimen with the complementary colour. Thus, red
colour results whenever green and blue is strongly absorbed.

Raman was also fascinated with colours due to pigments in
butterflies, and wanted to enlarge his collection by adding this
species of butterflies. Man of action that he was, he resorted to
catching butterflies. For this purpose, he would go to his country
estate in Kengeri and there, with a net bag attached to a long
pole, he would run after the butterflies as they flitted from one
branch to another. Bagging them is quite an easy operation, but
for a 65-year-old person to run after butterflies and bag them
is something unusual, at least in India. Raman was perhaps the
only Nobel Prize-winning physicist who chased butterflies!
He carried out this operation with great enthusiasm for several
weeks, often assisted by Padmanabhan.\index{Padmanabhan, J.} It seems to have amused
Lady Raman\index{Raman, Chandrasekhara Venkata!Raman, Lokasundari} very much, for she often jested about her husband
running after butterflies. The specimens were put in a jar and
brought to the Institute where Padmanabhan treated them and
mounted them in glass cases, as directed by Raman.

\newpage

\heading{Raman and the bees}
\addtocontents{toc}{\protect\contentsline{section}{Raman and the bees}{\thepage}}

\vskip .1cm

\index{Raman, Chandrasekhara Venkata!Traits/Interests}

\lhead[{\it\fontsize{9pt}{9pt}\selectfont\thepage}]{\it{\fontsize{9pt}{11pt}\selectfont Raman and the bees}}

\noindent
Since the Raman Research Institute\index{Raman Research Institute} is situated on a hillock
and, in the old days, was surrounded by agricultural land, the
palace orchard and the palace gardens, it was a haven for honey
bees. You could find beehives on the eastern side of the building,
beneath the roof hanging. Such beehives are to be seen even today
on the tower of the Indian Institute of Science.\index{Indian Academy of Sciences} These were wild
bees, deadly when provoked.

To the east of the main building, Raman had built his
Spectroscopic Laboratory. This was a stone building with a high
roof and had a dome structure built on it for a telescope. It was
Raman's idea to have an astronomical telescope, for he also had
a strong interest in astronomy. No telescope was installed there
during Raman's time. A back entrance and a staircase provided
access to the domed part of the building. The domed structure
also had a door which opened out onto an open terrace which
was actually the roof of the Spectroscopic Laboratory. The
beehives on the main building roof were at the same level as this
open terrace and only a short distance away.

One day, Raman was taking round an American visitor who
had been brought to the Institute by a Mr. Venkateswaran who
was the head of Associated Cements in Bangalore and known
to Raman. The three of them climbed to the open terrace above
the Spectroscopic Laboratory and Raman was apparently pointing
out the beehives some 100 ft. away when, suddenly, a swarm of
bees began to attack the party. They ran through the door leading
to the stairs, but, in the confusion, did not close the door behind
them. The angry bees pursued them and delivered their stings
without mercy. The American gentleman, in confusion and pain,
must have shaken his arm so violently, an expensive Rolex gold
watch flew from his wrist somewhere along the route. The three
of them, including Raman, kept running for their lives, shrieking
and uttering cries of pain, while the bees continued to swarm
around them. Venkateswaran sought protection in the parked
car near the portico and shut himself in. The servants of the
Institute, who witnessed the spectacle, ran for their lives as the
bees began to attack everyone around. A few of those attacked
ran to the water storage tanks located in the garden and
submerged themselves in the water to avoid the bees.

At the time, I was in the photographic dark room developing
some negatives. Raman sought refuge there, entering the room
turbanless and crying in pain. Evidently he had lost his turban
in the commotion and this only made matters worse, for the bees
venomously attacked his head. However, the bees left Raman
alone, once he entered the dark room, for they could not see in
the dark. But when I went out momentarily, I was stung a few
times; it was only then that I realised what had happened. After
a brief rest in the dark room, I took Raman to another large room
for fresh air. Seating him in a comfortable chair, I examined him.
I pulled out half a dozen bees from his ears with a forceps and
several stings from his head, face and hands. Raman was
exhausted; he was panting breathlessly and perspiring profusely.
I gave him some water, rested him and then had him taken home.

As for the American gentleman, the attack was a disaster
for him. Every exposed part of his body had swelled up and turned
black and blue. He fell unconscious on the lawn and had to be
hospitalised. Venkateswaran, meanwhile, took off in the car and
disappeared from the scene for a while. The next day, Raman
was back to normal and came to the Institute declaring, ``The bee
bite has made me stronger''. He was then about 66 or close to
it, and it is amazing how quickly he recovered. He had a robust
constitution and was tough.

After this incident, Raman declared war on the bees and
forthwith ordered their extermination. Padmanabhan\index{Padmanabhan, J.} was put
in charge of this operation and he used chemicals and fire to
destroy the beehives. The bees, however, were persistent and
returned to the same place after some time. But Padmanabhan
took necessary steps again to deal with them. It was a continuous
battle.

\medskip
\heading{Back to the starting point}
\addtocontents{toc}{\protect\contentsline{section}{Back to the starting point}{\thepage}}
\smallskip

\lhead[{\it\fontsize{9pt}{9pt}\selectfont\thepage}]{\it{\fontsize{9pt}{11pt}\selectfont Back to the starting point}}

\noindent
In the course of 1950s, all the students who had worked with
Raman obtained their Doctorates. Chandrasekhar\index{Chandrasekhar, S.} was selected
for the 1851 Exhibition scholarship and left for Cambridge in
1954. Later, Bhat\index{Bhat, M.R.} left for the U.S.A. under a Fullbright grant
and took his Ph.D. in physics at Ohio State University. In 1956,
Ramdas left to take up a post-doctoral position at Purdue
University in the U.S.A. Soon after Ramdas left, Krishnamurti,\index{Krishnamurti, D.}
Viswanathan,\index{Viswanathan, K.S.} Pancharatnam\index{Pancharatnam, S.} and myself were all appointed as
Assistant Professors at the Raman Research Institute,\index{Raman Research Institute} on a pay
scale of 600-50-1000. With free housing, it was not a bad situation.
We all continued with our work, which involved helping Raman
whenever he wanted us to and, during the rest of the time,
pursuing those of our studies approved as worthwhile by Raman.

\vskip .1cm

Some time in the early part of 1960, the Vice-Chancellor of
Mysore University requested Raman to recommend a few
physicists for the physics faculty in the newly-created postgraduate
centre at Mysore. Raman felt that he should only recommend
the very best men, especially those in whom he had the highest
confidence. Accordingly, he considered recommending Krishnamurti,\index{Krishnamurti, D.} 
Viswanathan\index{Viswanathan, K.S.} and Chandrasekhar\index{Chandrasekhar, S.} who was by then getting
ready to return from England. He mentioned all this to me and
said, ``I could also recommend you, but I want to have you as
well as Pancharatnam\index{Pancharatnam, S.} with me. This is a good opportunity for
the rest of them''.

\vskip .1cm

I left the Raman Research Institute\index{Raman Research Institute} in October 1960 for the
University of California,\index{University of California} Los Angeles, to work at the Institute
of Geophysics there with Prof. George C. Kennedy\index{Kennedy, George, C.} in the field
of High Pressure research. Raman was very unhappy over my
decision to leave his Institute and was very upset that I wrote
to Kennedy without his knowledge. Unfortunately, I had to do
this in a sneaky fashion because 1 knew how Raman would have
reacted if I had told him about it beforehand.

\vskip .1cm

For many years he had been telling me that he was sick of
people taking a Ph.D. and using it as a passport to go abroad.
1 had approached him many times for his permission to submit
some of my studies towards a Ph.D. degree. He would always
remark, ``Look here, I don't attach any importance to degrees
and you see I never took a Ph.D. in my life. Why do you want
a Ph.D.? Whether you have a Ph.D. or not, it does not make
any difference to my estimation of you''. Usually, after this
argument from him, I would not press my point. On one occasion,
he further elaborated what was in his mind. He said, ``You see
you will get a Ph.D. and then leave me, to go to some laboratory
abroad''. I perfectly understood his viewpoint; he did not want
this scenario to happen in my case. Both of us got on extremely
well and, in some sense, I had become somewhat indispensable
to him in his mind. My affection, loyalty and respect for him
were something extraordinary and I don't remember doing
anything that displeased him in any way.

At some point early in 1960 Raman finally agreed that I could
submit a Ph.D. thesis if I was so keen about it. I was by then
getting a little bit restless at the Institute and, having spent more
than a decade there, I wanted a change. It was marvellous to be
associated with a great scientist and be under his protection.
However, I began to feel that I should go out and make it
on my own. Hence I wrote to a friend, Dr. Sourirajan,\index{Sourirajan} who was
at that time working in Kennedy's laboratory at UCLA. I had
read some of Bridgman's\index{Bridgman} papers and had developed some interest
in the field of High Pressure research. Sourirajan encouraged
me to write to George Kennedy\index{Kennedy, George, C.} and I acted on his suggestion.
Within a fortnight I had an offer of a post-doctoral research
fellowship, with the title Assistant Research Geophysicist, to
work in Kennedy's laboratory, even though I had not yet obtained
my Ph.D.

Kennedy\index{Kennedy, George, C.} had suggested in his letter that I could get a Ph.D.
at UCLA while working on my research assignment. This was
the letter I showed to Raman. He got very upset and angry with
me. I explained to him that I would like to have a different
experience and see a little bit of the world, that this desire
prompted me to write to Kennedy. I told him that after spending
two years in California, I would return to the Institute. He said,
``No such thing. If you leave, you don't come back to my
Institute''. I told him, ``Sir, if that is your decision I feel myself
very unfortunate, but I have decided to go''. Then he read and
reread Kennedy's letter and said, ``All right, you submit your
Ph.D. thesis and then go. But you see, you can't come back here
and I don't know what you will do when you return''. I really
felt miserable that I would be leaving the Institute forever and that
too after having earned the displeasure of my mentor and
revered guru.

In the next few months I wrote my Ph.D. dissertation and
submitted it to the University of Madras. I left Bangalore on
October 12, 1960 for California. The fact that I might be
becoming one of those stereotypes he had talked about bothered
me very much. But the time had come for us to part company
and, with a heavy heart and tears in my eyes, I took leave of
my beloved Professor. For eleven years I had been a close
companion and confidant to him and he had, in turn, treated
me with utmost kindness and consideration. He made me a
research scientist and a physicist. I learned from him how to
appreciate nature, see loveliness and beauty in things and
the methodology of research. Many times he had told me,
``If you are interested in a subject, start your own study on it.
You will definitely find something new that others have missed.
It is the application of a keen and observant mind that is
important. You can do the literature survey later''. This has
proved to be so true for me in many of my studies.

After I left in October 1960 he was very upset and
recommended all his research team, including Pancharatnam,\index{Pancharatnam, S.} to
the Mysore faculty. Apparently he offered Pancharatnam a
Professorship at the Raman Institute which, I was told later,
Pancharatnam declined to accept. They were all appointed by
the Mysore University, but later there was a court case connected
with the appointments when some aggrieved party alleged
favouritism. I believe Raman had to appear in court and give
evidence. Finally, everything turned out all right, but this turn
of events was very unfortunate because Raman wanted to do the
best for the University and got hurt in the process. All the persons
involved represented the best men with him at the time and there
was no question of any favouritism.

After all this, Raman became very cynical in his attitude and
refused to take anyone into the Institute. With the assistance of
\hbox{Padmanabhan}\index{Padmanabhan, J.} and Balakrishnan he continued to pursue his
scientific interests until his death in 1970.

I visited India in 1964 and called on Raman. I could feel
that he was still upset with me for having left him, but I walked
with him in the garden for some time and told him about my
work in High Pressure research. He had grown weaker, but his
health was still good for his age.

Raman maintained his health remarkably well until a few
months before his death. His walks, dietary habits and means
of relaxation helped him to keep good health. When I joined him
in 1949 he was 61 years old and was in very good shape. He had
evidently developed hernia some years before and used to wear
a belt for this, but around 1952 the condition bothered him so
much he had to undergo surgery for it. Lady Raman\index{Raman, Chandrasekhara Venkata!Raman, Lokasundari} took him
to Vellore American Mission Hospital and had the surgery
performed by the well-known American surgeon Dr. Sommervel,\index{Sommervel}
who was at that time serving at the Mission Hospital. Raman
was a difficult patient. Lady Raman told us that she had a tough
time keeping him in bed while recuperating and had to report
the matter to Sommervel. The doctor apparently had to admonish
Raman, telling him that if he did not follow his instructions,
he (Raman) would croak and he (Sommervel) wouldn't be able
to do anything. Raman seems to have followed instructions after
that and, after a ten-day stay in the hospital, returned to
Bangalore. The hernia problem disappeared after the surgery and
Raman kept good health thereafter, except for minor illnesses.

Lady Raman used to bring a Dr. Subba Rao,\index{Rao, Subba} the resident
doctor at the Indian Institute of Science,\index{Indian Institute of Science} to treat Raman for any
minor illness. Raman liked Subba Rao very much, for the latter
would go along with his wishes.

\medskip
\heading{The founder's wish}
\addtocontents{toc}{\protect\contentsline{section}{The founder's wish}{\thepage}}
\smallskip

\lhead[{\it\fontsize{9pt}{9pt}\selectfont\thepage}]{\it{\fontsize{9pt}{11pt}\selectfont The founder's wish}}

\index{Raman Research Institute|(}

\noindent
Raman had bequeathed all his personal wealth to his
Research Institute and desired a bright future for it. He was very
much against accepting grants from the Government, for he
feared that would destroy the freedom necessary for carrying
out fundamental research. When the Education Minister,
M.C. Chagla,\index{Chagla, M.C.} once offered the Government of India's support,
Raman said, ``Sir, I want this Institute to be an oasis in the desert,
free from government interference and the application of its rules
and regulations. That would destroy my Institute. Thank you
for your offer''.

Raman had very definite ideas about how a research institute
should come into existence, how it should be managed and what
kind of persons should work therein. During his lifetime, he
expressed his views about such matters on several occasions, but
what he stated only a few days before his death is worth
reproducing here. This, almost in his own words, was recorded
by one of his associates:
\begin{quote}
{\fontsize{10}{12}\selectfont
``The Raman Research Institute was created by me in 1948
to provide a place in which I could continue my studies in an
atmosphere more conducive to pure research than that found in
most scientific institutions. To me the pursuit of science has been
an aesthetic and joyous experience. The Institute has been to me
a haven where I could carry on my highly personal research work.
This personal character of the Raman Research Institute should
obviously change after me. It must blossom into a great centre
of learning, embracing many branches of science. Scientists from
different parts of India and from all over the world must be
attracted to it. The foundations of such a centre have already been
laid. With its beautiful gardens, large libraries and extensive
museums, I feel that the Institute offers a perfect nucleus for the
growth of a centre of higher learning.

``I have always felt that science can only flower when there
is an internal urge. It cannot thrive under external pressures.
I strongly believe that fundamental science cannot be driven by instructional, industrial, Governmental or military pressures. 
This was the reason why I decided, as far as possible, not to accept
money from Government. I am a very practical man and I am
practical enough to see that it would not be possible for others
to run or grow a good institution without funds. I have bequeathed
all my property to the Institute. Unfortunately, this may not be
sufficient for the growth of this centre of learning. I, therefore,
will not put it as a condition that no Governmental funds should
be accepted by the Institute; I would, however, strongly urge
taking only funds that have no strings attached.

\newpage

``The full potential of this centre of learning can only be
achieved by exceptional leadership. Among the many qualities
called for in a person who assumes this responsibility are scientific
integrity, vision, receptiveness to new ideas and an enlightened
outlook, to let younger people grow unhindered to their full
height. If these qualities can also be combined with the scientific
reputation acquired by significant personal contributions in a
chosen field of endeavour, one has a leader who is likely to succeed
in developing the Institute and incidentally rendering the nation
a service. Any person who assumes the responsibility of running
the Institute must have full control of the laboratories, libraries,
workshops and other facilities. He must be empowered to acquire
and dispense money in the name of the Institute. He must have
powers to appoint or terminate the services of personnel required
for the running of the Institute. Nothing is so detrimental to the
growth of Science in an institution than the existence of dead-wood floating aimlessly, unable to participate in the scientific
growth of an Institute.''}\relax
\end{quote}

It is quite apparent that, in his own mind, Raman regarded
the Institute as a place for his work during his lifetime and, after
that, as a legacy to the succeeding generations of scientists in
India. On one occasion, when he was provoked by a news
reporter, he said, pointing towards his laboratory, ``This Institute
is a monument to my egotism. I am an egotist, and just as
the Egyptian Kings used to build pyramids before their death,
so is this Institute my pyramid''. Although this statement, taken
out of context, appears to give prominence to the ego in Raman,
it is no more than an expression of the uncommon degree of self-confidence 
which Raman always displayed in his own methods
of work. On the same occasion, he went on to say, ``You know
I was in the Indian Institute of Science\index{Indian Institute of Science} and was due to retire at
60. So, two years before my retirement, I started building this
Institute so that on the day I retired I took my bag and walked
into this Institute. I cannot remain idle for a single day''. That was
the main purpose for which the Raman Research Institute was
built by Raman and, indeed, it served the purpose so fully and
so admirably that it enabled Raman to work there almost every
day of the twenty-and-odd years that he lived after retirement
from his formal position at the Indian Institute of Science.

Raman\index{Raman, Chandrasekhara Venkata!Awards/Distinctions} was appointed as National Professor for life (the first
such appointment) by the Government of India, after his retirement 
from the Indian Institute of Science, to enable him to pursue
his interests at his new Institute. There is a story that a feeler
was sent to him to find out if he would accept the Vice-Presidency
of India. He is reported to have had a hearty laugh and said,
``What would I do with it?'' Raman was never a member of many
committees and he resigned in later years from even the few he
was connected with. He even resigned his Fellowship of the Royal
Society. All that mattered to him in the last two years was the
pursuit of his scientific interest and the future of the Raman
Research Institute.
\index{Raman Research Institute|)}

\bigskip
\heading{The last years}
\addtocontents{toc}{\protect\contentsline{section}{The last years}{\thepage}}
\smallskip

\lhead[{\it\fontsize{9pt}{9pt}\selectfont\thepage}]{\it{\fontsize{9pt}{11pt}\selectfont The last years}}

\noindent
Isolation from other scientists in India, arising partly from
his disappointment with the trends that the growth of Science
in India was showing, and partly from his desire to devote himself
wholly to his chosen lines of work, was a noticeable feature of
the last few years of Raman's life. He was generally critical of
post-Independence scientific developments in India and became
more strongly so as time went on. He bitterly complained against
the growing dependence of Indian scientists on foreign institutions
for their equipment and support and even for their ideas.
He disapproved of young men going out of India to build
scientific careers. However, during the last two decades of his
life, the times were such that the so-called `brain-drain' was
gaining momentum in India as in other developing countries.


He disapproved of organisations spending large sums of\break
money on equipment and often said that where there is creativity
of mind, the magnitude of external tools does not matter. But the
expansion in independent India was such that large sums of money
came to be invested on National Laboratories and other
Government-controlled scientific institutions. Thus, the widening
gap between his approach to Science and the way India's scientific
development was going alienated him from the mainstream and
he became somewhat cynical in his attitude. When the late
Jawaharlal Nehru,\index{Nehru, Jawaharlal} then Prime Minister of India, admonished
India's scientists and asked them to come out of the ivory towers
in which they had confined themselves, Raman reacted in typically
sharp manner and said, ``The men who matter are those who
sit in ivory towers. They are the salt of the earth and it is to them
that humanity owes its existence and progress''.


Many persons used to write to him and ask him for his views
on matters of general interest.\index{Raman, Chandrasekhara Venkata!Traits/Interests} During the last years of his life,
only rarely could a comment be elicited from him on a topic which
was not related in some way or the other to Science, at least in
its wider perspective. To one such query from an Indian scientist
working in the United States, Raman gave a reply in a letter dated
June 16, 1964, in the following words: ``My personal philosophy
of life about which you wish me to write is sufficiently indicated
by the facts of my career. My first scientific paper was published
in the {\em Philosophical Magazine}\index{Philosophical Magazine@\textit{Philosophical Magazine}} of London in November 1906 when
I was just 18 years of age. I am now over 75 years and do not
recollect any time during this long period when I took my mind
off from my scientific interests. Today I am as active as ever.''
This reply contains statements which are typical of Raman's
attitude to life.

To celebrate Raman's 80th birthday, the late Vikram
Sarabhai\index{Sarabhai, Vikram} organised the annual meeting of the Indian Academy
of Sciences\index{Indian Academy of Sciences} in 1968 at the Physical Research Laboratory in
Ahmedabad. The meeting was held early in December. I specially
came to India from the U.S.A. on this occasion, to see Raman
and to felicitate him. Raman was in excellent spirits and moved
freely with the scientists gathered there, joking and laughing with
them. I presented Raman with a collection of synthetically-grown
crystals from Bell Labs and synthetic diamonds grown at General
Electric Research Laboratories, Schenectady. F.P. Bundy\index{Bundy, F.P.} and
W.H. Wentorf,\index{Wentorf, W.H.} who had kindly given the diamonds to me, had
arranged them on a base to look like the letter R and had set
a plastic lens over them, so that the letter could be seen with the
naked eye. When I presented the collection to him, with some
appropriate remarks, Raman was visibly moved. Later, during
the conference, he seems to have remarked to someone,
``Jayaraman knows how to touch my heart. Unfortunately I am
unable to do anything for him in return that is good enough to
attract him back to the country''.


There was an evening dinner on the lawns of the Physical
Research Laboratory to felicitate Raman. Seated with him at the
centre table were a galaxy of Indian scientists, both Raman's old
students as well as other distinguished scientists. At the end of
the dinner, there were speeches in which a number of scientists
paid tributes to Raman, highlighting his achievements and
the important role he had played, directly or indirectly, in moulding their
 scientific careers. I remember distinctly that when
G.N. Ramachandran\index{Ramachandran, G.N.} tried to speak, he was so overcome by
emotion he broke down and could not proceed beyond one or
two opening sentences, before sitting down. Finally, it was
Raman's turn to reply. 
\begin{figure}[H]
\begin{center}
\includegraphics[scale=.95]{eps/11.eps}
\end{center}
{\fontsize{10pt}{11pt}\selectfont{\em C.V. Raman and Vikram Sarabhai during the Indian Academy of Sciences annual meeting held in Ahmedabad in December 1968. The Academy honoured Raman on his 80th birthday at this meeting.}}\relax
\end{figure}

\newpage

\begin{figure}[H]
\rotatebox{90}{
\begin{tabular}{c}
\\
\includegraphics[scale=1.2]{eps/13.eps}\\
{\fontsize{10pt}{12pt}\selectfont\em Group photo taken at the 1968 Annual Meeting of IASc in Ahmedabad. Celebrating Raman's 80th birthday.}
\end{tabular}}\relax
\end{figure}

He started by saying, ``You know, people may be wondering
why I wear a turban in this day and age. I will tell you why. The
turban is a bandage to prevent my getting a swollen head after
hearing such speeches...''. He looked at the sky, the stars and
the lovely trees around. He started talking about the wonder of
Nature, the excitement which comes from a search for truth and
the humility that a true scientific pursuit instils in one. He
remarked that there was so much to be studied and understood
that he felt that he had not accomplished anything worthwhile
in his life. He referred to Pascal's saying, ``Knowledge is
like a sphere in space; the greater its volume, the greater is its
contact with the unknown''. He went on, ``You have all spoken
in praise of my work and achievements, but I am not satisfied
with what I have done. What am I, compared to great scientists
like Einstein!''\index{Einstein, Albert} Raman referred to the enormous opportunities
which modern biological research had opened up for an
understanding of the meaning and mechanism of life processes.
The impromptu remarks he made after dinner on that occasion
left an indelible impression on everyone present.

Towards the end, he became an institution in himself and,
as loneliness surrounded him, work became all that mattered to
him in life. When he fell ill, and was confined to bed, the end
coming nearer and nearer, be told his doctors, ``I do not want
to survive my illness if it means anything less than a hundred
per cent active and productive life''. Less than a couple of months
before his death, he went up to the first floor of the Raman
Research Institute, seemingly as active as a young schoolboy, and
delivered the Gandhi Memorial lecture on October 2, 1970. It was
the last lecture he gave in his life and consisted of a masterly
exposition of his ideas about the theory of hearing, once again
illustrating his breadth of interests. This incident is more than
proof that he not only believed in work, but also practiced what
he believed in, by keeping himself active till the very end.


In October 1970 I arrived in Bangalore to spend a sabbatical
year and to set up facilities for High Pressure research in India.
My first duty was to pay my respects to my guru. In December
1968, when I saw him in Ahmedabad, he became tired very easily.
In the intervening period, between January 1969 and October
1970, he had suffered setbacks in health and was not his old self.
I saw him in his bungalow briefly and he got out of bed to talk
to me, in spite of Lady Raman's\index{Raman, Chandrasekhara Venkata!Raman, Lokasundari} protests that he should not do
so. That was, unfortunately, to be my last conversation with him.
In a few days he went into hospital after a heart attack.
His condition appeared to improve at first, but within a few
weeks he died. He passed away in the early hours of the morning
of Saturday, November 21, 1970.\index{Raman, Chandrasekhara Venkata!Death} By a special arrangement, his
mortal remains were cremated in the grounds of the Raman
Research Institute.\index{Raman Research Institute} Thousands of people, schoolchildren, students
and others from all walks of life thronged the Institute precincts
to pay homage to the memory of a great man.

Death is inevitable, but you don't want it to come to someone
who changed your life and made it worthwhile. I was in silent
tears that my revered guru had passed away, but it was some
consolation that I could be present when his soul was consigned
to eternal peace. I would have felt very badly had I missed paying
my homage and last respects to Chandrasekhara Venkata Raman,
that legendary figure of Science in modern India.

After Raman passed away, his younger son, V.  Radhakrishnan,\index{Radhakrishnan, V.} an eminent Radio Astronomer, took over as the
Director of the Institute. In the last 16 years, the Institute has
grown beyond recognition and is bustling with research activity
in liquid crystals, astrophysics and radio astronomy. The budget
has grown beyond anything Raman would have imagined and
the Institute is now largely supported by grants from the
Government of India.



