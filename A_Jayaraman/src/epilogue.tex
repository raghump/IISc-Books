\chapter*{Epilogue}
\addtocontents{toc}{\protect\contentsline{section}{\protect Epilogue}{\thepage} } 

\lhead[{\it\fontsize{9pt}{9pt}\selectfont\thepage}]{\it{\fontsize{9pt}{11pt}\selectfont Epilogue}}

Raman's life has been an extraordinary one in many ways.
But his most significant act in that extraordinary life was that
he took to the pursuit of Science, against all conventional wisdom
prevalent in his time. What prompted him to do so is obvious
in hindsight. He had an irresistible passion for physics, and the
creative spirit in him drove him to seek the opportunity to make
his dreams in that discipline come true. However, the destiny of
an individual is the result of interaction between his free will and
external circumstances. The latter are more often opportunities
that come by chance, over which the individual has no control.
In almost every person's life these elements play an important
role. In Raman's life, his entry into Government Service gave
him the financial independence to pursue his interest in physics
and his posting to Calcutta gave him the unique opportunity of
discovering a scientific institution suited to his temperament.

At that time, Calcutta had the best intellectual and cultural
traditions in India. Science had taken root there as an important
intellectual pursuit. On this, Raman has said, ``Whether a great
populous city offers the most suitable environment for the pursuit
of scientific research may well be questioned. Many instances may
be cited which seem unfavourable to the supposition. That the
centre of gravity of Science in Great Britain is to be found at
Cambridge and not in London or Edinburgh is probably no
accident. But Paris is a typical example of a great city which is
not only the political and social but also the intellectual capital
of its country. Calcutta claims a similar privilege so far as
Bengal is concerned, but an impartial observer would probably
also conclude without hesitation that the proud privilege she once
enjoyed of being the Imperial Capital has not yet disappeared
in the sphere of scientific activity. She owes her prestige and
influence in the sphere of learning to her centuries old tradition
of culture and research, to the long line of eminent scholars, both
Indian and European, whom Calcutta had and has in numbers
among her citizens, and not least to the efforts of such men as
the late Dr. Mahendra Lal Sircar\index{Sircar, Mahendra Lal} and Sir Asutosh Mookerjee,\index{Mookerjee, Asutosh}
who strove to create the facilities for higher studies and research
that others now enjoy''.

But for this intellectual atmosphere, there would not have
been an Indian Association for Cultivation of Science\index{The Indian Association for Cultivation of Science} to provide
Raman with an opportunity to renew his interest in physics.
Raman's discovery of the existence of the Association was
accidental. The open arms with which he was welcomed into it
were the encouragement he needed. Both were important turning
points in his life. Further, the Association was just the right body
for Raman's personality, because he functioned at his best
when he had all the independence and freedom to act and make
the decisions.

Raman possessed an indomitable spirit and an abiding love
of Science. He demonstrated by his example that independent
thinking, hard work, self-confidence and utter dedication are
absolute necessities for scientific achievement. Raman's scientific
intuition guided him to choose important problems that could
be tackled with the meagre facilities available and he made a mark
in whatever area he chose to investigate with these facilities.
From 1907 to 1917 \hbox{Raman} worked very hard, for his official
work, as Assistant Accountant General, was considerable; he
could carry out his scientific studies at the Indian Association
for Cultivation of Science only during his spare time. Yet he
energetically built up laboratory facilities there, conducted
experiments and published papers.

The success of his research programmes in acoustics and
optics fuelled his enthusiasm and confidence so much that, in
ten years, confident of further scientific successes, he was ready
to give up his lucrative position in the Financial Civil Service.
With his \hbox{appointment} to the Palit Professorship, Raman's
transformation to a full-time scientific Life became complete.

Realising that research in optics and light-scattering was the
area in which great discoveries could be made, he switched from
acoustics, although his successes in the latter field were spectacular.
 Of this he wrote in 1968, ``My studies on bowed string
instruments represent a phase of my earliest activities as a man
of Science. They were mostly carried out between the years 1914
and 1918. My call to the professorship at the Calcutta University
in July 1917 and the intensification of my interest in optics inevitably 
called a halt to my further studies of the violin family instruments.'' This was again a turning point in his scientific interest.

His innate aesthetic approach to Science, and his acceptance
of the older Lord Rayleigh\index{Rayleigh, Lord} as his model, led him to light-scattering
experiments. These he assiduously conducted with his collaborators
 for eight years. With perseverance, critical evaluation, and
step-by-step improvements in the methods of study, he made a
profound discovery that brought him a lasting reputation as a
great scientist. What is amazing is that Raman became an
 experimentalist {\em par excellence}, although he had received very little
training in college. Even today, the training in Indian universities
tends to be more theoretically oriented. It was more so during
Raman's time.

The scientific spirit is manifested primarily as a curiosity
about Nature and a deep desire to understand natural phenomena.
Experimentation, observation and interpretation constitute the
methodology of Science. Perseverance and dedication are its
operational requirements. By his life and work, Raman showed
how a true scientist should think, function and act. Never for
a moment did he lose his interest in Science, despite the
disappointments and frustrations he had to face in his lifetime.

Raman did not have any false modesty. In fact, he was
a supreme egotist, who was susceptible to flattery. At times
he exhibited uncontrollable anger and was even very rude in
his remarks. He was often criticised for these shortcomings.
But to persons who knew him well, his qualities as a great
scientist and lover of Nature \hbox{overshadowed} these shortcomings.
His enthusiasm, his simplicity and his directness touched anyone
who came near him.

\bigskip
\heading{Some important dates in the life of C.V. Raman}
\addtocontents{toc}{\protect\contentsline{section}{Some important dates in the life of C.V. Raman}{\thepage}}
\smallskip

\lhead[{\it\fontsize{9pt}{9pt}\selectfont\thepage}]{\it{\fontsize{9pt}{11pt}\selectfont Some important dates in the life of C.V. Raman}}

\begin{longtable}{@{}lcp{6cm}<{\raggedright}@{}}
1888, November 7 & --- &  Born at Thiruvanaikkaval near
 Tiruchchirappalli\\
1892-1902 & --- & Early education at Vishakhapatnam\\
1900 & --- & Matriculation Examination\\
1902 & --- & F .A. Examination, joins Presidency
 College, Madras\\
1904 & --- & B.A., 1st Rank, Gold Medal\\
1906 & --- & First paper published in {\em Phil. Mag.}, London\\
1907 & --- & M.A.; Financial Civil Service Examination, 1st Rank\\
& --- &  Married Lokasundari\\
& --- & Posted as Assistant Accountant-General,
Indian Finance Department, Calcutta\\
& --- & Starts working at the Indian Association
for Cultivation of Science (IACS), Calcutta\\
1907-1917 & --- & Officer, Finance Department, at
Calcutta, Rangoon, Nagpur, Calcutta\\
1917, July & --- & Palit Professor of Physics, Calcutta
University\\
1919, November & --- & Secretary, IACS\\
1921 & --- & First visit abroad (to England)\\
1924 & --- & Elected Fellow of the Royal Society, London\\
1928, February 28 & --- & Discovery of the Raman Effect in
Calcutta at the IACS\\
1928, March 16 & --- & First public lecture on the Raman Effect
before the South Indian Science Association at Central College, Bangalore\\
1929 & --- & Knighthood of the British Government\\
1930 & --- & Nobel Prize for Physics\\
& --- & Hughes Medal of the Royal Society\\
1933, March 31 & --- & Director, Indian Institute of Science, Bangalore\\
1934 & --- & Indian Academy of Sciences started by Raman\\
1935-36 & --- & Raman-Nath Theory: Diffraction of light by ultrasonic waves\\
1937 & --- & Resigns the Directorship of I.I.Sc., but continues as Professor \& Head, Department of Physics\\
1940 & --- & Raman-Nedungadi discovery of the soft mode behaviour in quartz\\
1942 & --- & Franklin Medal\\
1948, July & --- & Retires from I.I.Sc.\\
& --- & Raman Research Institute established.\\
& --- & Appointed National Research Professor\\
1954 & --- & Bharat Ratna\\
1957 & --- & International Lenin Peace Prize of the Soviet Union\\
1961 & --- & Member, Pontifical Academy of Sciences, The Vatican\\
1970, November 21 & --- & Passed away in Bangalore
\end{longtable}

