\chapter*{Foreword}
\addtocontents{toc}{\protect\contentsline{section}{\protect Foreword --- A. Jayaraman}{\thepage}}
 

I am very happy that this book about Professor C.V\@. Raman, who founded the Indian Academy of Sciences in 1934, is being reprinted and republished by the Academy this year. The first edition of this book was published in 1988 by the Affiliated East-West Press and is now out of print. This edition has been entirely retyped and reformatted to take advantage of the changes in publishing methods in the past thirty years. I would like to record my sincere thanks to the President and the Staff of the Indian Academy of Sciences for taking a keen interest in reprinting this book. I am grateful to Mrs Dominique Radhakrishnan and the Raman Research Institute for permission to reproduce some of the new photographs that have been included in this edition, and to Dr\@. Mohan Narayanan and Dr\@. Vinodh Narayanan for their help. 

The Indian Academy of Sciences is the right conduit to propagate the spirit in which this book has been written: I hope that it will impress upon young minds (in India and elsewhere) that great discoveries are the results of a keen observation of Nature and the persistent pursuit to find the truth. Raman used very simple equipment to make several significant discoveries using sunlight, including the Raman Effect itself. After the advent of lasers a Raman spectrum can be simply recorded in a matter of minutes,  but it was a major challenge in Raman's time. During his Calcutta years he turned to sunlight to carry out the light scattering experiments which culminated in the discovery of the Raman effect. Using a complementary set of dye filters, he isolated narrow regions of the Sun's spectrum and visually observed the very weak scattered light that had undergone a change in the frequency by interacting with molecules. Having seen with his own eyes that a very weak glow was separated by a dark band from the strong incident light, he employed a mercury arc lamp to confirm the nature of the very weak scattered light, namely that it consists of narrow lines, modified by the vibrating molecule. He has said many times that the cost of the equipment to make the discovery was a mere 500 Rupees of those times. I don't think many realize this - as well as the fact that it was scattered sunlight that gave him the first clue!

During the early years at the Raman Research Institute we did not have electricity for a long time -- it took eighteen months to complete electrification after my joining, when I went to work with Raman. Undeterred by this disadvantage Raman asked me to direct a beam of sunlight reflected by a mirror from outside to do some beautiful work on optical phenomena exhibited by a class of mineral samples that he had collected for display in his Museum as well as for research. I did not realize at that time that the Raman effect was discovered using sunlight until one day he demonstrated it to me while we were doing other experiments.  

Raman's was a very inspiring and colourful personality, and his contributions to science in India will last forever. I hope that this book captures these different aspects of Raman -- the man and the scientist. 

\bigskip
\bigskip


\hfill {\em A. Jayaraman}\quad~\,

\hfill
{\em Phoenix, February 2017.}


\chapter*{Foreword}
\addtocontents{toc}{\protect\contentsline{section}{\protect Foreword --- A.K. Ramdas}{\thepage}}

The lives of men and women who achieve great distinction in science, literature or art hold a great fascination for the general public. A creative accomplishment standing well above even superior excellence is awe-inspiring. Even great discoverers reach the height of creativity only a few times during their career; the intense emotion then felt by them is best illustrated by story of Archimedes running through the streets of Syracuse shouting ``Eureka''!

The contemporary style of scientific reporting, as it has evolved, however, leaves out much of the drama --- the initial inspiration, the feverish pursuit, the false trails and frustrations, and, finally, the\break ecstasy of discovery. Even the presentation of scientific controversies in the professional journals is subdued, thanks to alert editors! The excitement of scientific research, the colourful persons who populate the scientific community, the clash of personalities --- all of these are carefully excluded from published literature, only to become a part of scientific folklore. Biographies of great personalities in science are, therefore, all the more precious documents. When written by their contemporaries, especially if by one of their close associates, they are invaluable.

Sir C.V\@. Raman, the discoverer of the Raman effect, made\break numerous innovative and original contributions to modern physics, optics and acoustics during the first half of this century. By his scientific accomplishments and by his unique scientific leadership in modern India, he made a profound impact. Dr\@. A\@. Jayaraman, a condensed matter physicist internationally known for his pioneering\break contributions to the physics of matter subjected to ultra-high pressures, was closely associated with Raman when he founded the Raman\break Research Institute in 1949 and developed it into a great centre of research.

For over eleven years, Jayaraman had the opportunity to interact with Raman on almost a daily basis and observe his scientific style, philosophy and motivations. We thus have here an authoritative\break biography of the most unique scientific personality of modern India, written with genuine understanding and admiration but with critical judgment and honesty which do not avoid the discussion of the all\break too human limitations of a great man.  The result is an authentic\break narration from a person who knows his subject. Jayaraman has\break enriched the history of science by writing this biography.

 
\bigskip
\bigskip
\begin{flushright}
\begin{tabular}{c@{}}
{\em A.K. Ramdas}\\
{\em Professor of Physics}\\
{\em Purdue University}
\end{tabular}
\end{flushright}

\bigskip

\noindent
7th February 1989\\
West Lafayette, IN 47907\\
U.S.A.
